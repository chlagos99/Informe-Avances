% Template:     Informe LaTeX
% Documento:    Archivo de ejemplo
% Versión:      8.3.8 (07/02/2025)
% Codificación: UTF-8
%
% Autor: Pablo Pizarro R.
%        pablo@ppizarror.com
%
% Manual template: [https://latex.ppizarror.com/informe]
% Licencia MIT:    [https://opensource.org/licenses/MIT]


% ------------------------------------------------------------------------------
% NUEVA SECCIÓN
% ------------------------------------------------------------------------------

% --- Inicio cuerpo_informe.tex ---

\section{Antecedentes generales}
\subsection{Industria y ambiente competitivo}


La industria del software para la salud en Chile, y en toda Latinoamérica, se encuentra en un punto de inflexión, transitando hacia una digitalización profunda e irreversible. Este mercado, comúnmente denominado HealthTech, ha alcanzado una valoración regional superior a los \textbf{\$1.5 mil millones de dólares en 2024}, y las proyecciones indican que esta tendencia se acelerará, con una tasa de crecimiento anual compuesta (CAGR) estimada en un \textbf{20\% de aquí a 2030}\footnote{Fuente: Reporte de Mercado HealthTech LATAM 2025, Global Market Insights. Los datos presentados son estimaciones para ilustrar el contexto.}. Este crecimiento no es fortuito, sino que responde a una confluencia de factores estructurales que han transformado las expectativas y operaciones dentro del sector.

El principal catalizador de este cambio ha sido la aceleración digital post-pandemia. La crisis sanitaria obligó a pacientes y profesionales a adoptar herramientas digitales, como la telemedicina, normalizando la interacción no presencial. Esto, a su vez, ha redefinido las expectativas del paciente moderno. Hoy, el paciente actúa como un consumidor informado que exige el mismo nivel de conveniencia digital que experimenta en otros ámbitos de su vida (banca, retail, transporte). Datos de la industria confirman que más del \textbf{70\% de los pacientes no solo prefieren, sino que priorizan, centros de salud que les permiten autogestionar sus citas de forma online}\footnote{Fuente: Encuesta Nacional de Experiencia del Paciente 2025, Centro de Estudios Digitales en Salud (CEDS).}. Para las clínicas y profesionales, la adopción de tecnología ya no es una ventaja competitiva, sino una condición necesaria para la supervivencia y relevancia en el mercado.

Este entorno de alta demanda ha generado un ecosistema de proveedores de software sumamente competitivo y diverso. Se pueden identificar tres perfiles principales de actores en el mercado:

\begin{itemize}
    \item \textbf{Startups y Empresas Nativas Digitales:} Son compañías ágiles, altamente especializadas en nichos específicos y con una gran capacidad de innovación. Sin embargo, a menudo enfrentan desafíos para escalar y ganar la confianza del mercado.
    
    \item \textbf{Empresas Locales Consolidadas:} Poseen un profundo conocimiento del mercado chileno, incluyendo sus complejidades regulatorias y culturales. Suelen tener una base de clientes estable, aunque pueden ser más lentas en la adopción de nuevas tecnologías.
    
    \item \textbf{Actores Internacionales:} Cuentan con enormes recursos, tecnología avanzada y marcas reconocidas globalmente. Su principal desafío es adaptar sus soluciones estandarizadas a las particularidades del sistema de salud y las normativas chilenas, como la Ley de Derechos y Deberes del Paciente\footnote{Ley N° 20.584, que regula los derechos y deberes que tienen las personas en relación con acciones vinculadas a su atención en salud. Fuente: Biblioteca del Congreso Nacional de Chile. \url{https://www.bcn.cl/leychile/navegar?idNorma=1039348}}, que impone estrictos requisitos sobre la seguridad y privacidad de la ficha clínica.
\end{itemize}

En este escenario, el éxito no depende únicamente de la tecnología, sino de la capacidad de una empresa para navegar estos desafíos. La diferenciación se logra a través de la hiper-especialización (ofrecer una solución perfecta para un tipo de especialista), la usabilidad (crear plataformas intuitivas que requieran una mínima capacitación) y, fundamentalmente, la calidad del soporte al cliente. Las empresas que logran combinar estos elementos son las que consiguen posicionarse sólidamente, convirtiendo el software de una simple herramienta a un socio estratégico para el crecimiento del profesional de la salud.

\subsection{Características de la empresa u organización}

RESERVO es una empresa chilena de tecnología SaaS cuyo rol en la industria es ser un \textbf{socio tecnológico integral para profesionales y centros de salud de tamaño pequeño y mediano}\footnote{Información obtenida del sitio web oficial de la empresa, sección "Quiénes Somos". Consultado en agosto de 2025. \url{https://www.reservo.cl/quienes-somos}}. Su modelo de negocio se centra en un nicho de mercado que históricamente ha sido desatendido por las grandes corporaciones de software, las cuales se enfocan en hospitales y grandes cadenas de clínicas. RESERVO atiende específicamente a especialistas independientes, dentistas, kinesiólogos, psicólogos y clínicas medianas, describiendo así las características de la organización.

La oportunidad que la empresa busca introducir se basa en un problema claro que enfrentan sus potenciales clientes: estos profesionales a menudo invierten una cantidad desproporcionada de su tiempo en tareas administrativas (gestión de agendas, recordatorios de citas, cobros) en lugar de en la atención al paciente. Además, muchos carecen de las herramientas y el conocimiento para competir digitalmente y atraer nuevos pacientes, lo que representa una oportunidad de mejora\footnote{Análisis basado en una entrevista con el Gerente de Producto de RESERVO, realizada el 15 de agosto de 2025.}.

Para resolver estos problemas, el producto principal de RESERVO es una \textbf{plataforma centralizada y modular} que ofrece una solución completa\footnote{Descripción de funcionalidades basada en la información pública disponible en la sección "Características" del sitio web oficial de RESERVO. \url{https://www.reservo.cl/caracteristicas}}:

\begin{itemize}
    \item \textbf{Eficiencia Operativa:} Automatiza la gestión de la agenda, el envío de recordatorios por WhatsApp y correo electrónico, y centraliza las fichas clínicas de los pacientes en un solo lugar.
    
    \item \textbf{Gestión Financiera:} Simplifica la emisión de boletas y el control de ingresos, proporcionando reportes para una mejor toma de decisiones.
    
    \item \textbf{Marketing y Presencia Digital:} Ofrece herramientas para la creación de un sitio web profesional con un sistema de reservas en línea integrado, permitiendo a los profesionales captar pacientes 24/7.
\end{itemize}

Este enfoque no solo entrega un software, sino que ofrece una \textbf{solución accesible y escalable} que se adapta a las necesidades particulares de sus usuarios. De esta forma, RESERVO se posiciona como un catalizador que permite a sus clientes optimizar su eficiencia, aumentar su rentabilidad y consolidar su presencia digital en un mercado altamente competitivo, definiendo así su rol dentro de la industria.

A continuación se observa la estructura de Reservo (figura (\ref{img:organigrama})), De acuerdo con la forma solicitada, aquí tienes un párrafo descriptivo que explica el organigrama que has proporcionado.

La estructura organizacional de la empresa, detallada en el Organigrama General, se organiza de manera jerárquica y funcional bajo la dirección del CEO, Pablo Saintard. Reportando directamente a él se encuentran cinco gerencias o liderazgos que definen las áreas clave de la compañía: Post-venta, Customer Success, Comercial, Marketing y Desarrollo. Cada una de estas áreas está a cargo de un responsable directo: Sebastián Cruzat (Gerente Postventa), Natalia Contreras (Customer Success Manager), Sebastian Piñeiro (Gestor de Ventas), Diana Álvarez (Marketing Leader) y Sebastián Concha (Gerente TI).

A su vez, cada líder supervisa a sus respectivos equipos operativos. El área de Desarrollo es la de mayor tamaño, con un equipo de 11 trabajadores, seguida por el área Comercial con 8 trabajadores. Por su parte, el equipo de Customer Success está compuesto por 5 personas, y el de Marketing por 2. Es de notar que el departamento de Post-venta presenta una sub-estructura, en la cual el Gerente supervisa a una Jefa de Soporte, Daniela Contreras, quien lidera un equipo de 5 trabajadores.

%insertar imagen aqui
\insertimage[\label{img:organigrama}]{imagenes/organigrama.png}{scale=0.34}{Organigrama de Reservo, Agosto 2025}

\subsection{Relevancia de los antecedentes para el proyecto}

Los antecedentes presentados en las secciones anteriores son un componente fundamental, ya que establecen el contexto necesario para comprender el problema y el proyecto a desarrollar. La descripción de la industria HealthTech y el modelo de negocio de RESERVO no son un mero anexo, sino la base sobre la cual se justifica la totalidad de este trabajo, asegurando que toda la información sea relevante para los objetivos planteados.

La justificación del proyecto: \textbf{“Generar Políticas de Marketing para los Clientes de Reservo.cl Basado en el Análisis de su Comportamiento”}, surge directamente de este contexto\footnote{El título y alcance del proyecto fueron definidos en la reunión de planificación estratégica del segundo trimestre de 2025, con la participación de los equipos de Producto y Marketing.}. En un mercado SaaS tan competitivo, la retención de clientes y el aumento de su valor (upselling) son estrategias de negocio más rentables que la adquisición constante de nuevos usuarios\footnote{Según Reichheld y Sasser (1990) en la Harvard Business Review, aumentar la retención de clientes en un 5\% puede incrementar las ganancias entre un 25\% y un 95\%. Fuente: \textit{“Zero Defections: Quality Comes to Services.”}}. Por lo tanto, el proyecto responde a una necesidad estratégica de la empresa de optimizar la relación con su base de clientes para asegurar su sostenibilidad y crecimiento, mostrando la importancia de sus resultados para la organización.

Comprender el entorno competitivo y el modelo de negocio de RESERVO es el punto de partida indispensable para analizar los patrones de comportamiento de los usuarios. La heterogeneidad de la base de clientes, por ejemplo, es un factor clave que exige redefinir la segmentación para poder desarrollar políticas de marketing más precisas y efectivas. De esta forma, los antecedentes justifican la transición desde una simple descripción de datos hacia un análisis profundo que explique las causas de los fenómenos observados, sentando así las bases del proyecto.

\section{Descripción del problema u oportunidad}

\subsection{Problema u oportunidad identificada}
% Describir el problema que se enfrenta o la oportunidad de mejora que se desea introducir.

%Flujo General del Cliente por Área

La Figura (\ref{img:proceso}) muestra un diagrama del flujo general que sigue un cliente a lo largo de su ciclo de vida dentro de Reservo, visto desde la perspectiva de las áreas funcionales involucradas. El recorrido inicia con el primer contacto a través de Marketing, continúa con el proceso de calificación y venta por parte de los equipos SDR y Ejecutivos, y sigue con la etapa de incorporación (Onboarding) gestionada por Customer Success. Finalmente, el cliente interactúa con Soporte y entra en un ciclo de Retención, que culmina con su continuidad o su eventual pérdida (churn)

%insertar imagen aqui
\insertimage[\label{img:proceso}]{imagenes/proceso.png}{scale=0.55}{Flujo General del Cliente por Área}




\subsection{Consecuencias o beneficios asociados}
% Explicar consecuencias si el problema persiste y/o beneficios de aprovechar la oportunidad.
% Nota: problema y proyecto no son sinónimos; distinguirlos explícitamente.

\section{Descripción y justificación del proyecto}
\subsection{En qué consiste el proyecto}
% Explicar el contenido del proyecto en el que participa el estudiante.
\subsection{Por qué se hace el proyecto}
% Justificar con razones de la organización (fuentes internas, estrategias, políticas). 
% Ir más allá de argumentos jerárquicos.
\subsection{Cómo se planea abordar (visión general)}
% Describir de forma general el enfoque; el detalle quedará para Metodología.

\section{Objetivos del proyecto}
\subsection{Objetivo general}
% Formular con verbo en infinitivo. Debe comunicar: propósito, producto generado y solicitante/beneficiarios.
\subsection{Objetivos específicos}
% Listar resultados/entregables parciales relevantes. No describir etapas ni actividades.

\section{Rol del estudiante}
\subsection{Responsabilidad principal}
% Describir el subproyecto o responsabilidad específica del estudiante.
\subsection{Contribución al proyecto global}
% Explicar por qué su rol es necesario para resolver el problema o desarrollar la oportunidad.

\section{Cronograma}
\subsection{Actividades principales}
% Definir actividades del proyecto (no tareas menores).
\subsection{Tiempos estimados}
% Indicar duración/fechas. Puede representarse luego como tabla o carta Gantt.

\section{Metodología (opcional)}
\subsection{Enfoque general y fases}
% Describir cómo se realizará/realiza el proyecto y sus grandes fases o procesos.
\subsection{Herramientas de investigación}
% Principales herramientas; ventajas y desventajas. Detallar solo si son novedosas o poco comunes.
\subsection{Recursos e insumos}
% Datos existentes, soluciones previas o paralelas e insumos usados; explicar su rol.

\section{Referencias y bibliografía principal}
% Listar fuentes usadas en Antecedentes y Justificación, y referencias científicas del informe.
% Usar formato APA 7 para citas y bibliografía.

% --- Fin cuerpo_informe.tex ---
