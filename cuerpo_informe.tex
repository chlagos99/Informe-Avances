% Template:     Informe LaTeX
% Documento:    Cuerpo de Informe
% Versión:      8.3.8 (07/02/2025)
% Codificación: UTF-8
%
% Autor: Pablo Pizarro R.
%        pablo@ppizarror.com
%
% Manual template: [https://latex.ppizarror.com/informe]
% Licencia MIT:    [https://opensource.org/licenses/MIT]


% ------------------------------------------------------------------------------
% NUEVA SECCIÓN
% ------------------------------------------------------------------------------

% --- Inicio cuerpo_informe.tex ---

% ------------------------------------------------------------------------------
% NUEVA SECCIÓN
% ------------------------------------------------------------------------------
\section{Antecedentes generales}
\subsection{Industria y ambiente competitivo}


El sector de la salud en Chile y en toda Latinoamérica está experimentando una transformación digital irreversible. Este mercado, conocido como HealthTech o salud digital, alcanzó una valoración de \textbf{\$17\,mil millones de dólares en 2024} en toda la región y se proyecta que su crecimiento se acelere. Las estimaciones indican que el mercado podría alcanzar los \textbf{\$58\,mil millones de dólares para 2030}, lo que representa una \textbf{tasa de crecimiento anual compuesta (CAGR) del 23.2\,\%} durante este periodo \citep{GrandViewResearch2024}. Este dinamismo no es casual, sino que es impulsado por la creciente adopción de tecnologías digitales, el aumento de la demanda de telemedicina y la necesidad de mejorar la eficiencia en la atención sanitaria \citep{MordorIntelligence2025}.


El principal catalizador de este cambio en Latinoamérica y Chile ha sido la aceleración digital impulsada por la pandemia, que expandió rápidamente la telemedicina y sentó las bases de un modelo híbrido de atención \citep{SuperSalud2025}. El impulso post COVID-19 redefinió expectativas: pacientes y equipos valoran la continuidad de cuidados, el acceso a especialistas a distancia y la coordinación entre niveles, pero también se observan asimetrías de recursos y brechas de adopción que condicionan el uso sostenido \citep{RevChilSalud2025}. En atención primaria, la estrategia de telesalud se ha usado para gestionar demanda y acceso, apoyando la resolución oportuna y la articulación con especialidades \citep{PanAmPubHealth2024}. Con este contexto, para prestadores y clínicas en Chile, la adopción efectiva de tecnologías —desde teleconsulta hasta autogestión de citas y canales asincrónicos— deja de ser ventaja y se vuelve condición para mantener relevancia y calidad asistencial \citep{SuperSalud2025}.

Este entorno de alta demanda ha generado un ecosistema de proveedores de software sumamente competitivo y diverso. Se pueden identificar tres perfiles principales de actores en el mercado:

\begin{itemize}
    \item \textbf{Startups y Empresas Nativas Digitales:} Son compañías ágiles, altamente especializadas en nichos específicos y con una gran capacidad de innovación. Sin embargo, a menudo enfrentan desafíos para escalar y ganar la confianza del mercado.
    
    \item \textbf{Empresas Locales Consolidadas:} Poseen un profundo conocimiento del mercado chileno, incluyendo sus complejidades regulatorias y culturales. Suelen tener una base de clientes estable, aunque pueden ser más lentas en la adopción de nuevas tecnologías.
    
    \item \textbf{Actores Internacionales:} Cuentan con enormes recursos, tecnología avanzada y marcas reconocidas globalmente. Su principal desafío es adaptar sus soluciones estandarizadas a las particularidades del sistema de salud y las normativas chilenas, como la Ley de Derechos y Deberes del Paciente \citep{Ley20584}, que impone estrictos requisitos sobre la seguridad y privacidad de la ficha clínica.

\end{itemize}

En este escenario, el éxito no depende únicamente de la tecnología, sino de la capacidad de una empresa para navegar estos desafíos. La diferenciación se logra a través de la hiper-especialización (ofrecer una solución perfecta para un tipo de especialista), la usabilidad (crear plataformas intuitivas que requieran una mínima capacitación) y, fundamentalmente, la calidad del soporte al cliente. Las empresas que logran combinar estos elementos son las que consiguen posicionarse sólidamente, convirtiendo el software de una simple herramienta a un socio estratégico para el crecimiento del profesional de la salud.

\subsection{Características de la empresa u organización}

RESERVO es una empresa chilena de tecnología SaaS cuyo rol en la industria es ser un \textbf{socio tecnológico integral para profesionales y centros de salud de tamaño pequeño y mediano} \citep{ReservoQuienesSomos2025}. Su modelo de negocio se centra en un nicho de mercado que históricamente ha sido desatendido por las grandes corporaciones de software, las cuales se enfocan en hospitales y grandes cadenas de clínicas. RESERVO atiende específicamente a especialistas independientes, dentistas, kinesiólogos, psicólogos y clínicas medianas, describiendo así las características de la organización.

La oportunidad que la empresa busca introducir se basa en un problema claro que enfrentan sus potenciales clientes: estos profesionales a menudo invierten una cantidad desproporcionada de su tiempo en tareas administrativas (gestión de agendas, recordatorios de citas, cobros) en lugar de en la atención al paciente. Además, muchos carecen de las herramientas y el conocimiento para competir digitalmente y atraer nuevos pacientes, lo que representa una oportunidad de mejora\footnote{Análisis basado en una entrevista con la Customer Success Manager de RESERVO, realizada el 15 de agosto de 2025.}.

Para resolver estos problemas, el producto principal de RESERVO es una \textbf{plataforma centralizada y modular} que ofrece una solución completa\footnote{Descripción de funcionalidades basada en la información pública disponible en la sección "Características" del sitio web oficial de RESERVO. \url{https://www.reservo.cl/caracteristicas}}:

\begin{itemize}
    \item \textbf{Eficiencia Operativa:} Automatiza la gestión de la agenda, el envío de recordatorios por WhatsApp y correo electrónico, y centraliza las fichas clínicas de los pacientes en un solo lugar.
    
    \item \textbf{Gestión Financiera:} Simplifica la emisión de boletas y el control de ingresos, proporcionando reportes para una mejor toma de decisiones.
    
    \item \textbf{Marketing y Presencia Digital:} Ofrece herramientas para la creación de un sitio web profesional con un sistema de reservas en línea integrado, permitiendo a los profesionales captar pacientes 24/7.
\end{itemize}

Este enfoque no solo entrega un software, sino que ofrece una \textbf{solución accesible y escalable} que se adapta a las necesidades particulares de sus usuarios. De esta forma, RESERVO se posiciona como un catalizador que permite a sus clientes optimizar su eficiencia, aumentar su rentabilidad y consolidar su presencia digital en un mercado altamente competitivo, definiendo así su rol dentro de la industria.

La estructura organizacional de la empresa, detallada en el Organigrama General (Figura \ref{img:organigrama}), se organiza de manera jerárquica y funcional bajo la dirección del CEO, Pablo Saintard. Reportando directamente a él se encuentran cinco gerencias o liderazgos que definen las áreas clave de la compañía: Post-venta, Customer Success, Comercial, Marketing y Desarrollo. Cada una de estas áreas está a cargo de un responsable directo: Sebastián Cruzat (Gerente Postventa), Natalia Contreras (Customer Success Manager), Sebastian Piñeiro (Gestor de Ventas), Diana Álvarez (Marketing Leader) y Sebastián Concha (Gerente TI).

A su vez, cada líder supervisa a sus respectivos equipos operativos. El área de Desarrollo es la de mayor tamaño, con un equipo de 11 trabajadores, seguida por el área Comercial con 8 trabajadores. Por su parte, el equipo de Customer Success está compuesto por 5 personas, y el de Marketing por 2. Es de notar que el departamento de Post-venta presenta una sub-estructura, en la cual el Gerente supervisa a una Jefa de Soporte, Daniela Contreras, quien lidera un equipo de 5 trabajadores.

%insertar imagen aqui
\insertimage[\label{img:organigrama}]{imagenes/organigrama.png}{scale=0.34}{Organigrama de Reservo, Agosto 2025. Fuente: elaboración propia}


% ------------------------------------------------------------------------------
% NUEVA SECCIÓN
% ------------------------------------------------------------------------------
\section{Descripción del problema u oportunidad}

\subsection{Antecedentes}
% Describir el problema que se enfrenta o la oportunidad de mejora que se desea introducir.

%Flujo General del Cliente por Área

\subsubsection{Proceso general}
La Figura \ref{img:proceso} muestra un diagrama del flujo general que sigue un cliente a lo largo de su ciclo de vida dentro de Reservo, visto desde la perspectiva de las áreas funcionales involucradas. El recorrido inicia con el primer contacto a través de Marketing, continúa con el proceso de calificación y venta por parte de los equipos SDR y Ejecutivos, y sigue con la etapa de incorporación (Onboarding) gestionada por Customer Success. Finalmente, el cliente interactúa con Soporte y entra en un ciclo de Retención, que culmina con su continuidad o su eventual pérdida (churn)

%insertar imagen aqui
\insertimage[\label{img:proceso}]{imagenes/proceso.png}{scale=0.65}{Flujo General del Cliente por Área. Fuente: elaboración propia}

%Estados del cliente / Ciclo de vida del cliente en reservo

\subsubsection{Etapa de Captación}

El proceso de captación, ilustrado en el anexo~\ref{img:captacion}, es iniciado por el área de \textbf{Marketing}, que ejecuta una acción específica para atraer el interés de potenciales clientes. Un \textbf{cliente} potencial encuentra esta acción de marketing y procede a registrar su información en la plataforma de Reservo. Este registro representa la primera transición crucial, cambiando el estado del \textbf{cliente} de un anónimo \textbf{`Desconocido'} a un \textbf{`Lead'} conocido dentro del sistema de Reservo.

Una vez enviados los datos, estos son recibidos por el equipo de \textbf{Ventas SDR} (Sales Development Representative), que realiza la tarea inicial de configurar el lead para su posterior contacto. El \textbf{SDR} procede a contactar al prospecto; en este punto, el flujo depende de la respuesta del \textbf{cliente}, ya que el proceso solo continúa si este contesta y acepta seguir adelante. Tras una interacción exitosa, el \textbf{SDR} ejecuta la tarea de `Calificar Lead' para determinar si el prospecto encaja con el perfil de cliente ideal de la empresa. Si la calificación es positiva, el \textbf{cliente} experimenta su segundo cambio de estado importante, convirtiéndose en un \textbf{`Lead calificado'}.

Para concluir esta fase, el equipo \textbf{SDR} se encarga de agendar una demostración con un \textbf{ejecutivo de ventas} y de enviar una cotización formal. La programación exitosa de esta demo marca la tercera y última transición de estado en este diagrama, elevando al prospecto a una \textbf{`Oportunidad de Venta'}. Con esto finaliza oficialmente el proceso de captación inicial y se transfiere al \textbf{cliente} a la siguiente etapa del embudo de ventas.

\subsubsection{Etapa de Ventas}

El proceso de Venta, detallado en el anexo~\ref{img:etapa_venta}, comienza cuando un \textbf{cliente} en estado de `Oportunidad de Venta' asiste a una demostración del producto. Una vez finalizada la demo, si el \textbf{cliente} decide contratar el servicio, el \textbf{Ejecutivo de Venta} procede a configurar al prospecto en el sistema, momento en el cual se produce la primera transición de estado importante, actualizando su condición a \textbf{`Precliente'}. A continuación, el \textbf{Ejecutivo de Venta} envía las instrucciones de pago y, una vez que el \textbf{cliente} completa dicha transacción, se realiza el cambio de estado final: el prospecto es convertido oficialmente en un \textbf{`Cliente'}. Finalmente, el \textbf{Ejecutivo de Venta} transfiere al nuevo \textbf{cliente} al área de \textbf{Customer Success}, dando inicio al `Proceso de Onboarding' y concluyendo así la etapa comercial.

\subsubsection{Etapa de Onboarding}

El proceso de Onboarding, como se detalla en el anexo~\ref{img:etapa_onboarding}, comienza inmediatamente después de que finaliza la venta, con el equipo de \textbf{Customer Success} preparando la cuenta e invitando al nuevo \textbf{cliente} a una primera capacitación. Una vez realizada esta sesión inicial, inicia una fase de seguimiento para monitorear si el \textbf{cliente} utiliza activamente la plataforma. Durante este período, ocurren las transiciones más relevantes, ya que, basándose en el nivel de riesgo detectado, un \textbf{cliente} puede ser clasificado como de \textbf{`riesgo alto'} o \textbf{`riesgo medio'}. Esta clasificación desencadena acciones de contacto proactivo por parte del equipo de \textbf{Customer Success}. Si el \textbf{cliente} no presenta riesgos y transcurre un período determinado, el proceso de Onboarding se da por finalizado, marcando la transición del \textbf{cliente} a un estado de uso regular de la herramienta.

\subsubsection{Etapa de Cliente Activo}

La etapa de \textbf{`Cliente Activo'}, ilustrada en el anexo~\ref{img:etapa_cliente_activo}, representa el ciclo de vida del \textbf{cliente} una vez finalizado el proceso de Onboarding, centrándose en la retención y expansión. En esta fase, el equipo de \textbf{Customer Success} realiza un monitoreo constante de la salud de la cuenta. Si se detecta un uso deficiente, se activan acciones de seguimiento, lo que implícitamente clasifica al \textbf{cliente} como un perfil \textbf{`en riesgo'} para mitigar una posible baja. Por el contrario, si la salud de la cuenta es buena, el equipo busca oportunidades de expansión (upselling); si el \textbf{cliente} acepta y completa el pago, su condición se actualiza, reafirmándolo como un \textbf{`cliente vigente'}. Paralelamente, el proceso gestiona las interacciones del \textbf{cliente} con el área de \textbf{Soporte} y los ciclos de facturación, donde un fallo recurrente en el pago puede culminar con la \textbf{pérdida del cliente (churn)}.

\subsubsection{Etapa de Cliente Inactivo}

El proceso de \textbf{`Cliente Inactivo'}, detallado en el anexo~\ref{img:etapa_cliente_inactivo}, gestiona tanto la baja de un \textbf{cliente} como su posible reactivación. El flujo se inicia cuando un \textbf{cliente} solicita cancelar su servicio, ya sea durante el Onboarding o en su ciclo de vida activo. El equipo de \textbf{Customer Success} procesa esta solicitud, formalizando la transición del \textbf{cliente} a un estado \textbf{`dado de baja'}. Posteriormente, se registran las razones de la pérdida y se informa al \textbf{cliente} para cerrar el ciclo. El diagrama también contempla un camino para la recuperación; si un \textbf{cliente} inactivo decide volver a contratar, el equipo de \textbf{Customer Success} gestiona el nuevo contrato y la reactivación de la cuenta. Esta acción revierte el estado del \textbf{cliente}, devolviéndolo a la condición de \textbf{`cliente activo'} y reintegrándolo al ciclo de vida principal.

\subsection{Problema encontrado}
Para asegurar su crecimiento en el competitivo mercado SaaS, Reservo debe optimizar continuamente el valor que extrae de su base de clientes. Actualmente, la organización enfrenta desafíos que limitan su capacidad para retener y expandir cuentas de manera eficiente. Estos desafíos se manifiestan en una serie de efectos observables que impactan directamente la rentabilidad y la sostenibilidad del negocio.

\subsubsection{Efectos y consecuencias observadas}
Los resultados de la gestión actual del ciclo de vida del cliente presentan tres consecuencias principales:

\begin{itemize}
    \item \textbf{Pérdida recurrente de clientes e ingresos:} La empresa experimenta una \textbf{`Tasa de Churn'} mensual del \textbf{2.46\%}, lo que se ha traducido en \textbf{70 bajas} en lo que va del año 2025\footnote{La \textbf{Tasa de Churn} se define en la tabla~\ref{tab:estados-cliente}. El valor y el número de bajas corresponden a los datos internos de la empresa para el año 2025. Fuente: Base de datos de Reservo.}. Esta fuga constante de clientes impacta directamente los ingresos recurrentes y frena el crecimiento neto del negocio. Para que la empresa pueda expandirse, los ingresos de nuevos clientes deben superar consistentemente el valor perdido por las bajas, lo que representa un desafío significativo para la escalabilidad.

    \item \textbf{Expansión y reactivación no sistemáticas:} Aunque la empresa logra generar valor adicional, los resultados no muestran un crecimiento acelerado. Durante 2024 se lograron \textbf{57 upsellings}, mientras que en 2025 se han conseguido \textbf{60} hasta la fecha. En cuanto a la reactivación, en 2024 se recuperaron \textbf{100 clientes}, mientras que en 2025 van \textbf{89}\footnote{Los datos de upsellings y clientes reactivados corresponden a los registros internos de la empresa. Fuente: Base de datos de Reservo.}. Estas cifras sugieren un crecimiento modesto en la expansión y una posible disminución en la capacidad de reactivar cuentas, indicando la falta de una estrategia proactiva y sistemática.

    \item \textbf{Asignación ineficiente de recursos de postventa:} El equipo de Customer Success opera sin una priorización clara de su cartera. Los esfuerzos se distribuyen de manera uniforme, sin distinguir entre clientes con alto riesgo de abandono y aquellos con alto potencial de expansión, lo que diluye el impacto de las acciones de retención.
\end{itemize}

\subsubsection{Análisis y Elección del Problema Central}
Los efectos descritos (pérdida de clientes, expansión no optimizada e ineficiencia) son síntomas de varios problemas subyacentes. Para seleccionar el problema central, se utiliza como base teórica el marco de \textbf{Gestión del Ciclo de Vida del Cliente (Customer Lifecycle Management)}, que postula que las acciones de una empresa deben adaptarse a la etapa y al comportamiento específico de cada cliente para ser efectivas.

Bajo esta óptica, la elevada tasa de abandono y el bajo crecimiento del valor no son problemas raíz, sino consecuencias directas de un problema más profundo y operativo. Se elige como problema central aquel que, de ser solucionado, tendría el mayor impacto en mitigar los demás. Por lo tanto, se descartan los síntomas y se selecciona la causa operativa fundamental que los origina. El problema central es:

\begin{center}
    \textit{Tratamiento no diferenciado de la cartera de clientes en la etapa de postventa.}
\end{center}

\subsubsection{Causas del Problema Central}
Este problema no se debe a una falta de datos, sino a la forma en que la empresa los interpreta y los utiliza para guiar sus acciones. Las causas fundamentales son:

\begin{itemize}
    \item \textbf{Toma de decisiones basada en indicadores de resultado:} Las estrategias se activan a partir de métricas retrospectivas, como la \textbf{`Tasa de Churn'} o un \textbf{`NPS'} de \textbf{52}\footnote{El \textbf{NPS} se define en la tabla~\ref{tab:estados-cliente}. El valor de 52 es el último resultado medido por la empresa. Fuente: Base de datos de Reservo.}. Estos indicadores miden un resultado que ya ocurrió, pero no permiten diferenciar el tratamiento entre clientes \textit{antes} de que el riesgo se materialice.

    \item \textbf{Análisis estático de la información:} La empresa analiza datos de manera estática, observando el estado actual de un cliente, pero no su evolución en el tiempo. Se carece de una visión longitudinal del comportamiento, lo que impide identificar tendencias, como una disminución gradual del uso de la plataforma, que son señales tempranas de un posible abandono.

    \item \textbf{Ausencia de criterios conductuales para la segmentación:} No se ha definido formalmente qué patrones de uso se correlacionan con un cliente saludable, en riesgo o con potencial de expansión. Sin estos criterios, es imposible segmentar la cartera de manera dinámica para aplicar acciones diferenciadas.

    \item \textbf{Enfoque operativo centrado en la reacción a eventos:} El flujo de trabajo del equipo de Customer Success está estructurado para reaccionar a eventos negativos (quejas, impagos) en lugar de estar diseñado para segmentar proactivamente la base de clientes y ejecutar acciones personalizadas.
\end{itemize}
\section{Descripción y Justificación del Proyecto}

La sección anterior definió como problema central el tratamiento no diferenciado de la cartera de clientes en la etapa de postventa, lo que genera consecuencias negativas en la retención, expansión y eficiencia operativa de Reservo. A continuación, se presenta un proyecto diseñado para abordar directamente las causas de este problema, sentando las bases para una gestión del cliente más estratégica y proactiva basada en evidencia teórica y mejores prácticas gerenciales.

\subsection{Definición y Alcance del Proyecto}

El presente proyecto, titulado \textbf{“Diseño de Políticas de Clientes de Reservo.cl, Basado en el Análisis de su Comportamiento”}, consiste en la creación de un marco analítico fundamentado en el análisis del comportamiento de los usuarios y el ciclo de vida del cliente.  

El entregable final no será la implementación de una herramienta de software, sino el diseño, la validación y la propuesta de dicho modelo analítico, junto con un conjunto de políticas y acciones recomendadas para que el equipo de Customer Success pueda aplicarlas a cada segmento identificado.

\subsection{Justificación Estratégica}

La realización de este proyecto se justifica por su alineación directa con la necesidad de la empresa de optimizar la rentabilidad de su cartera y asegurar un crecimiento sostenible. Diversos estudios señalan que la gestión proactiva del cliente basada en datos es un eslabón crítico para la competitividad y la rentabilidad en mercados digitales \citep{Reinartz2004}.

Las razones estratégicas se basan en tres pilares principales:  

\begin{itemize}
    \item \textbf{Mitigar la pérdida de ingresos por churn:} La tasa de churn es uno de los indicadores más relevantes para la salud financiera de empresas con modelos de suscripción. Reducirla suele ser más rentable que adquirir nuevos clientes, como destacan Gupta et al.\citep{Gupta2006}. Actualmente, la tasa de abandono mensual del \textbf{2.46\%} afecta directamente los ingresos recurrentes. Un modelo predictivo permitiría anticipar señales de deserción y activar acciones de retención personalizadas, en línea con la teoría del ciclo de vida del cliente\citep{Lemmens2008}.

    \item \textbf{Capitalizar oportunidades de expansión:} Más allá de la retención, es fundamental aumentar el valor de vida del cliente (CLV). Estrategias de upselling y cross-selling aplicadas de manera segmentada permiten sistematizar la expansión en lugar de dejarla como práctica reactiva \citep{Kumar2010}. Actualmente, la evolución modesta en upsellings (57 en 2024 frente a 60 en 2025) muestra un potencial poco explotado.

    \item \textbf{Optimizar el esfuerzo del equipo de Customer Success:} El trato homogéneo de la cartera genera sobrecostos y baja efectividad. Una segmentación basada en datos permitirá focalizar los recursos en clientes de mayor riesgo o mayor potencial, elevando la eficiencia y alineándose con las prácticas recomendadas en CRM estratégico \citep{Reinartz2004b}.

\end{itemize}



\newpage
\section{Objetivos del proyecto}

\subsection{Objetivo general}
% Formular con verbo en infinitivo. Debe comunicar: propósito, producto generado y solicitante/beneficiarios.

Diseñar un modelo analítico de segmentación conductual para la cartera de clientes de Reservo, que permita a la empresa diferenciar y gestionar de manera proactiva las estrategias de retención y expansión para cada tipo de cliente. Este enfoque se basa en los principios de segmentación conductual, que se centran en el conocimiento, la actitud y el uso que un cliente hace de un producto para definir las acciones de marketing \citep{Kotler2016}.

\subsection{Objetivos específicos}
% Listar resultados/entregables parciales relevantes. No describir etapas ni actividades.

Para alcanzar el objetivo general, se deben generar los siguientes resultados específicos, cada uno de los cuales aborda una de las causas del problema central:

\begin{itemize}
    \item \textbf{Identificar y validar un conjunto de indicadores de comportamiento} que sirvan como señales tempranas (leading indicators) para predecir el riesgo de abandono y las oportunidades de expansión, superando la dependencia actual de métricas de resultado (lagging indicators).

    \item \textbf{Definir y caracterizar al menos tres segmentos de clientes} basados en sus patrones de comportamiento dinámicos y su evolución en el tiempo, proporcionando así los criterios conductuales estandarizados que actualmente no existen.

    \item \textbf{Analizar la evolución temporal del comportamiento de los clientes} para identificar las trayectorias típicas que siguen los clientes antes de un evento crítico (como el churn o un upselling), abordando así la limitación del análisis estático actual.
    
    \item \textbf{Proponer un conjunto de políticas de negocio personalizadas} para cada segmento identificado, que guíen las acciones proactivas de los equipos de Marketing y Customer Success y permitan superar el actual enfoque operativo centrado en la reacción.
\end{itemize}

\section{Cronograma}
El plan de trabajo del proyecto se ha estructurado en un horizonte temporal de 18 semanas, abarcando el período comprendido entre agosto y diciembre de 2025. Para asegurar una ejecución metódica y el cumplimiento de los objetivos, el proyecto se ha dividido en cuatro fases secuenciales, cada una con entregables y actividades específicas.

\subsection{Actividades principales} %Ver si se puede agregar cosas del CRISP DM
La división del proyecto en fases permite una gestión estructurada del alcance y facilita el seguimiento de los avances. A continuación, se describen las actividades fundamentales de cada etapa:

\begin{itemize}
    \item \textbf{Fase 1: Planificación y Fundamentos.} Esta etapa inicial establece las bases conceptuales y operativas del proyecto. Comprende el levantamiento de requerimientos con los \textit{stakeholders}, la formalización del árbol de problemas, y la definición del alcance y los objetivos. Esta fase culmina con la validación de la línea base sobre la cual se medirá el éxito del proyecto.

    \item \textbf{Fase 2: Análisis de Datos y Desarrollo del Modelo.} Corresponde al núcleo analítico del trabajo. Se inicia con la recopilación, limpieza y pre-procesamiento de los datos históricos de la empresa. Posteriormente, se ejecuta un análisis exploratorio para identificar variables de comportamiento relevantes y, finalmente, se procede con el desarrollo y la validación iterativa del modelo de segmentación.

    \item \textbf{Fase 3: Diseño de la Propuesta Estratégica.} En esta fase, los resultados del modelo analítico se traducen en valor de negocio. Se realiza una caracterización detallada de cada segmento de cliente identificado y se diseña un portafolio de políticas de retención y expansión personalizadas. El trabajo concluye con la consolidación de todos los hallazgos en el informe final.

    \item \textbf{Fase 4: Cierre y Presentación Final.} La última etapa se dedica a la comunicación de los resultados. Se prepara el material de apoyo y se realiza la presentación final del proyecto, sus conclusiones y recomendaciones a la organización, junto con la entrega de toda la documentación y artefactos generados.
\end{itemize}

\subsection{Tiempos estimados}
La siguiente tabla presenta la distribución de las actividades en el tiempo, detallando las tareas específicas de cada fase y su duración estimada en semanas.

\begin{table}[H]
  \begin{threeparttable}
    \centering
    \scriptsize
    \caption{Cronograma Detallado de Fases y Actividades del Proyecto}
    \renewcommand{\arraystretch}{1.5}
    \begin{tabular}{|l|p{9cm}|c|}
      \hline
      \textbf{Fase} & \textbf{Actividades Detalladas} & \textbf{Duración (Semanas)} \\
      \hline
      \textbf{1. Planificación} & 
      - Levantamiento y validación de requerimientos. \newline
      - Definición del problema, objetivos y alcance. \newline
      - Elaboración del primer informe de avance. & 
      \textbf{4} \\
      \hline
      \textbf{2. Desarrollo del Modelo} & 
      - Recopilación y pre-procesamiento de datos. \newline
      - Análisis exploratorio de datos (EDA). \newline
      - Desarrollo y entrenamiento del modelo. \newline
      - Validación y ajuste de métricas de rendimiento. & 
      \textbf{9} \\
      \hline
      \textbf{3. Propuesta Final} & 
      - Caracterización de segmentos de clientes. \newline
      - Diseño de políticas de negocio personalizadas. \newline
      - Redacción del informe final. & 
      \textbf{4} \\
      \hline
      \textbf{4. Cierre} & 
      - Preparación del material de apoyo. \newline
      - Presentación final de resultados. & 
      \textbf{1} \\
      \hline
      \multicolumn{2}{|r|}{\textbf{Duración Total Estimada}} & \textbf{18} \\
      \hline
    \end{tabular}
    \label{tab:cronograma}
  \end{threeparttable}
\end{table}

\newpage

\section{}
% ------------------------------------------------------------------------------
% REFERENCIAS, revisar configuración \stylecitereferences \def\stylecitereferences {natbib} 
% ------------------------------------------------------------------------------
\clearpage
\bibliographystyle{apalike}
\bibliography{library}
% Listar fuentes usadas en Antecedentes y Justificación, y referencias científicas del informe.
% Usar formato APA 7 para citas y bibliografía.


% ------------------------------------------------------------------------------
% ANEXO
% ------------------------------------------------------------------------------
\newpage

\begin{appendixd}

	\section{Estados del cliente}

    \subsection{}
    \insertimage[\label{img:captacion}]{bizagi/captacion.png}{scale=0.27}{Etapa de Adquisición en Reservo. Fuente: elaboración propia}
    \subsection{}
    \insertimage[\label{img:etapa_venta}]{bizagi/venta.png}{scale=0.25}{Etapa de Venta en Reservo. Fuente: elaboración propia}
    \subsection{}
    \insertimage[\label{img:etapa_onboarding}]{bizagi/onboarding.png}{scale=0.19}{Etapa de Onboarding del cliente en Reservo. Fuente: elaboración propia}
    \subsection{}
    \insertimage[\label{img:etapa_cliente_activo}]{bizagi/cliente_activo.png}{scale=0.15}{Etapa de cliente activo en Reservo. Fuente: elaboración propia}
    \subsection{}
    \insertimage[\label{img:etapa_cliente_inactivo}]{bizagi/cliente_inactivo.png}{scale=0.30}{Etapa de Cliente inactivo en Reservo. Fuente: elaboración propia}

    	% Imagen, se numerará automáticamente con la letra del anexo según
	% la configuración \appendixindepobjnum
	  \insertimage[\label{img:proceso_completo}]{bizagi/proceso_completo.png}{scale=0.05}{Proceso completo de cliente en Reservo. Fuente: elaboración propia}

%Tabla con indicadores

  \section{Indicadores del proceso}
\begin{table}[H]
  \begin{threeparttable}
    \centering
    \scriptsize
    \caption{Métricas asociadas a clientes. Fuente: Elaboración propia}
    \renewcommand{\arraystretch}{1.3}
    \begin{tabular}{|l|C{3.5cm}|C{3.3cm}|C{5.5cm}|}
      \hline
      \textbf{Estado} & \textbf{Definición} & \textbf{Métricas asociadas} & \textbf{Fórmula / Cálculo} \\
      \hline
      Desconocido & Persona que aún no ha interactuado con la marca. & Visitas al sitio web. &
      Visitas web: nº de sesiones únicas registradas en la web. \\
      \hline
      Lead & Persona que mostró interés inicial y dejó sus datos de contacto. &
      Costo por Lead (CPL). &
      CPL = $\tfrac{\text{Gasto en marketing}}{\text{Nº leads generados}}$ \\
      \hline
      Lead calificado & Lead que ha sido validado por el equipo SDR como un cliente potencial. &
      Tasa de conversión a SQL (Sales Qualified Lead). &
      SQL rate = $\tfrac{\text{Nº leads calificados}}{\text{Nº leads totales}} \times 100$ \\
      \hline
      Oportunidad de venta & Lead calificado que ha aceptado una demostración o propuesta formal. &
      Win Rate. &
      Win rate = $\tfrac{\text{Nº ventas cerradas}}{\text{Nº oportunidades}} \times 100$ \\
      \hline
      Cliente nuevo & Ha contratado el servicio y se ha completado su alta en el sistema. &
      Costo de Adquisición de Cliente (CAC). &
      CAC = $\tfrac{\text{Costo total de Marketing y Ventas}}{\text{Nº clientes nuevos adquiridos}}$ \\
      \hline
      Onboarding (0--90 días) & Cliente nuevo en el proceso de implementación y capacitación inicial. &
      Tasa de Activación. &
      Activación = $\tfrac{\text{Nº clientes que completan onboarding}}{\text{Nº clientes nuevos}} \times 100$ \\
      \hline
      Cliente activo & Cliente que ha completado el onboarding y usa la plataforma regularmente. &
      Engagement Score. &
      Engagement = Frecuencia y profundidad de uso de funciones clave. \\
      \hline
      Cliente en riesgo & Cliente con bajo uso, quejas recurrentes o problemas de pago. &
      \% de Clientes en Riesgo. &
      \% en riesgo = $\tfrac{\text{Nº clientes en riesgo}}{\text{Nº clientes activos}} \times 100$ \\
      \hline
      Cliente en expansión & Cliente activo que contrata nuevas funcionalidades, módulos o sedes. &
      Net Dollar Retention (NDR). &
      NDR = $\tfrac{\text{MRR inicial + Expansión - Churn}}{\text{MRR inicial}} \times 100$ \\
      \hline
      Cliente Churn & Cliente que ha cancelado su suscripción y no renueva. &
      Tasa de Churn. &
      Churn Rate = $\tfrac{\text{Nº clientes perdidos}}{\text{Nº clientes al inicio del período}} \times 100$ \\
      \hline
      Valor de Vida del Cliente & Predicción del beneficio neto atribuido a toda la relación futura con un cliente. &
      Lifetime Value (LTV). &
      LTV = (Ticket promedio $\times$ Recurrencia) $\times$ Vida del cliente \\
      \hline
    \end{tabular}
    \label{tab:estados-cliente}
  \end{threeparttable}
\end{table}

\end{appendixd}

% --- Fin cuerpo_informe.tex ---
