% Template:     Informe LaTeX
% Documento:    Archivo de ejemplo
% Versión:      8.3.8 (07/02/2025)
% Codificación: UTF-8
%
% Autor: Pablo Pizarro R.
%        pablo@ppizarror.com
%
% Manual template: [https://latex.ppizarror.com/informe]
% Licencia MIT:    [https://opensource.org/licenses/MIT]


% ------------------------------------------------------------------------------
% NUEVA SECCIÓN
% ------------------------------------------------------------------------------

% --- Inicio cuerpo_informe.tex ---

\section{Antecedentes generales}
\subsection{Industria y ambiente competitivo}
% Contextualizar la industria y el entorno competitivo donde opera la organización.
\subsection{Características de la empresa u organización}
% Describir rasgos clave de la organización y su rol en la industria.
\subsection{Relevancia de los antecedentes para el proyecto}
% Incluir solo antecedentes útiles para comprender el contexto del problema y del proyecto.

\section{Descripción del problema u oportunidad}
\subsection{Problema u oportunidad identificada}
% Describir el problema que se enfrenta o la oportunidad de mejora que se desea introducir.
\subsection{Consecuencias o beneficios asociados}
% Explicar consecuencias si el problema persiste y/o beneficios de aprovechar la oportunidad.
% Nota: problema y proyecto no son sinónimos; distinguirlos explícitamente.

\section{Descripción y justificación del proyecto}
\subsection{En qué consiste el proyecto}
% Explicar el contenido del proyecto en el que participa el estudiante.
\subsection{Por qué se hace el proyecto}
% Justificar con razones de la organización (fuentes internas, estrategias, políticas). 
% Ir más allá de argumentos jerárquicos.
\subsection{Cómo se planea abordar (visión general)}
% Describir de forma general el enfoque; el detalle quedará para Metodología.

\section{Objetivos del proyecto}
\subsection{Objetivo general}
% Formular con verbo en infinitivo. Debe comunicar: propósito, producto generado y solicitante/beneficiarios.
\subsection{Objetivos específicos}
% Listar resultados/entregables parciales relevantes. No describir etapas ni actividades.

\section{Rol del estudiante}
\subsection{Responsabilidad principal}
% Describir el subproyecto o responsabilidad específica del estudiante.
\subsection{Contribución al proyecto global}
% Explicar por qué su rol es necesario para resolver el problema o desarrollar la oportunidad.

\section{Cronograma}
\subsection{Actividades principales}
% Definir actividades del proyecto (no tareas menores).
\subsection{Tiempos estimados}
% Indicar duración/fechas. Puede representarse luego como tabla o carta Gantt.

\section{Metodología (opcional)}
\subsection{Enfoque general y fases}
% Describir cómo se realizará/realiza el proyecto y sus grandes fases o procesos.
\subsection{Herramientas de investigación}
% Principales herramientas; ventajas y desventajas. Detallar solo si son novedosas o poco comunes.
\subsection{Recursos e insumos}
% Datos existentes, soluciones previas o paralelas e insumos usados; explicar su rol.

\section{Referencias y bibliografía principal}
% Listar fuentes usadas en Antecedentes y Justificación, y referencias científicas del informe.
% Usar formato APA 7 para citas y bibliografía.

% --- Fin cuerpo_informe.tex ---
