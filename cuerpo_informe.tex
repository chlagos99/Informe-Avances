% Template:     Informe LaTeX
% Documento:    Cuerpo de Informe
% Versión:      8.3.8 (07/02/2025)
% Codificación: UTF-8
%
% Autor: Pablo Pizarro R.
%        pablo@ppizarror.com
%
% Manual template: [https://latex.ppizarror.com/informe]
% Licencia MIT:    [https://opensource.org/licenses/MIT]


% ------------------------------------------------------------------------------
% NUEVA SECCIÓN
% ------------------------------------------------------------------------------

%               ------GUARDAR EN GITHUB------
%git add .
%git commit -m "Descripción de los cambios"
%git push
%
% --- Inicio cuerpo_informe.tex ---

% ------------------------------------------------------------------------------
% NUEVA SECCIÓN
% ------------------------------------------------------------------------------
\section{Antecedentes generales}
\subsection{Industria y ambiente competitivo}


El sector de la salud en Chile y en toda Latinoamérica está experimentando una transformación digital irreversible. En 2024 la CEPAL lanzó el Observatorio de Desarrollo Digital, que reúne \textbf{100+ indicadores en 12 áreas} y evidencia brechas: \textbf{más del 60\,\%} de las empresas que usan Internet mantienen presencia pasiva y cerca del \textbf{70\,\%} de las MIPYMES no tiene presencia en línea, reforzando la urgencia de digitalizar procesos de salud y servicios conexos \citep{CepalDDO2024}. En paralelo, el ecosistema regional de innovación en salud ya mapea un \textbf{universo de 1{,}000+ empresas/startups} —con focos relevantes en \textbf{prevención (232)} y \textbf{sistemas (225)}— según el reporte conjunto IDB Lab–HolonIQ \citep{IDBLabHolonIQ2024}. En Chile, el dinamismo del pipeline emprendedor se observa en Start-Up Chile: la cohorte BIG 4 incorporó \textbf{87 startups} (\textbf{20\,\%} en Health \& Biotech) y destina \textbf{más de \$2{,}5 mil millones CLP} en apoyos; desde 2010 la aceleradora ha respaldado \textbf{2{,}200+ startups}, con \textbf{\$5{,}8 mil millones USD} de valoración de portafolio y \textbf{\$2{,}1 mil millones USD} en ventas acumuladas \citep{StartUpChile2025}. El flujo de capital de riesgo hacia \textit{healthtech} en América Latina también creció con fuerza: de \textbf{\$16 millones USD (2019)} a \textbf{\$99 millones USD (2020)} y superando \textbf{\$1{,}000 millones USD en 2021} \citep{EmolHealthtech2023}. En este contexto, y siguiendo la \textit{Classification of Digital Health Interventions} de la OMS, la salud digital abarca intervenciones para \textit{clientes}, \textit{proveedores} y \textit{gestores del sistema}, cubriendo desde teleconsulta y autogestión hasta analítica y soporte a decisiones, lo que delimita el alcance de las soluciones consideradas en esta memoria \citep{WHOCDHI2018}.


El principal catalizador de este cambio en Latinoamérica y Chile ha sido la aceleración digital impulsada por la pandemia, que expandió rápidamente la telemedicina y sentó las bases de un modelo híbrido de atención \citep{SuperSalud2025}. El impulso post COVID-19 redefinió expectativas: pacientes y equipos valoran la continuidad de cuidados, el acceso a especialistas a distancia y la coordinación entre niveles, pero también se observan asimetrías de recursos y brechas de adopción que condicionan el uso sostenido \citep{RevChilSalud2025}.

Este entorno de alta demanda ha generado un ecosistema de proveedores de software sumamente competitivo y diverso. Se pueden identificar tres perfiles principales de actores en el mercado:

\begin{itemize}
    \item \textbf{Startups y Empresas Nativas Digitales:} Son compañías ágiles, altamente especializadas en nichos específicos y con una gran capacidad de innovación. Sin embargo, a menudo enfrentan desafíos para escalar y ganar la confianza del mercado.
    
    \item \textbf{Empresas Locales Consolidadas:} Poseen un profundo conocimiento del mercado chileno, incluyendo sus complejidades regulatorias y culturales. Suelen tener una base de clientes estable, aunque pueden ser más lentas en la adopción de nuevas tecnologías.
    
    \item \textbf{Actores Internacionales:} Cuentan con enormes recursos, tecnología avanzada y marcas reconocidas globalmente. Su principal desafío es adaptar sus soluciones estandarizadas a las particularidades del sistema de salud y las normativas chilenas, como la Ley de Derechos y Deberes del Paciente \citep{Ley20584}, que impone estrictos requisitos sobre la seguridad y privacidad de la ficha clínica.

\end{itemize}

En Chile, el universo direccionable para Reservo combina prestadores individuales habilitados y prestadores institucionales ambulatorios. A marzo de 2024, el Registro de la Superintendencia de Salud contabiliza \textbf{433.431 profesionales de la salud certificados} (concentrados en psicólogos 85.273, enfermeros 80.495 y médicos 59.624), que constituyen la base de potenciales cuentas individuales y consultas privadas \citep{ClinicasChile2025}. En el segmento institucional privado, a febrero de 2025 existían \textbf{555 prestadores privados acreditados}, de los cuales \textbf{175} corresponden a \textit{atención abierta} (centros ambulatorios)—el subsegmento con demanda directa por agendamiento, recordatorios y gestión de pacientes \citep{Superintendencia2025Acreditados}.

El espacio competitivo local y regional muestra escala medible. \textbf{AgendaPro} reporta \textbf{135.000 profesionales} y \textbf{20.000 negocios} gestionados, con más de \textbf{100 millones} de citas procesadas \citep{AgendaPro2025BusinessWire}. \textbf{Medilink} (Healthatom) informa \textbf{+7.000 clientes}, \textbf{+60.000 usuarios} y \textbf{42 millones} de citas efectivas \citep{MedilinkWeb2025}. \textbf{Encuadrado}, enfocada en profesionales independientes de salud, comunica \textbf{400 centros} en 2025 y una base superior a \textbf{5.000 profesionales} activos \citep{Encuadrado2025Centros}.



\subsection{Características de la empresa u organización}

RESERVO es una empresa chilena de tecnología SaaS cuyo rol en la industria es ser un \textbf{socio tecnológico integral para profesionales y centros de salud de tamaño pequeño y mediano} \citep{ReservoQuienesSomos2025}. Su modelo de negocio se centra en un nicho de mercado que históricamente ha sido desatendido por las grandes corporaciones de software, las cuales se enfocan en hospitales y grandes cadenas de clínicas. RESERVO atiende específicamente a especialistas independientes, dentistas, kinesiólogos, psicólogos y clínicas medianas, describiendo así las características de la organización.

La oportunidad que la empresa busca introducir se basa en un problema claro que enfrentan sus potenciales clientes: estos profesionales a menudo invierten una cantidad desproporcionada de su tiempo en tareas administrativas (gestión de agendas, recordatorios de citas, cobros) en lugar de en la atención al paciente. Además, muchos carecen de las herramientas y el conocimiento para competir digitalmente y atraer nuevos pacientes, lo que representa una oportunidad de mejora\footnote{Análisis basado en una entrevista con la Customer Success Manager de RESERVO, realizada el 15 de agosto de 2025.}.

Para resolver estos problemas, el producto principal de RESERVO es una \textbf{plataforma centralizada y modular} que ofrece una solución completa\footnote{Descripción de funcionalidades basada en la información pública disponible en la sección "Características" del sitio web oficial de RESERVO. \url{https://www.reservo.cl/caracteristicas}}:

\begin{itemize}
    \item \textbf{Eficiencia Operativa:} Automatiza la gestión de la agenda, el envío de recordatorios por WhatsApp y correo electrónico, y centraliza las fichas clínicas de los pacientes en un solo lugar.
    
    \item \textbf{Gestión Financiera:} Simplifica la emisión de boletas y el control de ingresos, proporcionando reportes para una mejor toma de decisiones.
    
    \item \textbf{Marketing y Presencia Digital:} Ofrece herramientas para la creación de un sitio web profesional con un sistema de reservas en línea integrado, permitiendo a los profesionales captar pacientes 24/7.
\end{itemize}

Este enfoque no solo entrega un software, sino que ofrece una \textbf{solución accesible y escalable} que se adapta a las necesidades particulares de sus usuarios. De esta forma, RESERVO se posiciona como un catalizador que permite a sus clientes optimizar su eficiencia, aumentar su rentabilidad y consolidar su presencia digital en un mercado altamente competitivo, definiendo así su rol dentro de la industria.

La estructura organizacional de la empresa, detallada en el Organigrama General (Figura \ref{img:organigrama}), se organiza de manera jerárquica y funcional bajo la dirección del CEO, Pablo Saintard. Reportando directamente a él se encuentran cinco gerencias o liderazgos que definen las áreas clave de la compañía: Post-venta, Customer Success, Comercial, Marketing y Desarrollo. Cada una de estas áreas está a cargo de un responsable directo: Sebastián Cruzat (Gerente Postventa), Natalia Contreras (Customer Success Manager), Sebastian Piñeiro (Gestor de Ventas), Diana Álvarez (Marketing Leader) y Sebastián Concha (Gerente TI).

A su vez, cada líder supervisa a sus respectivos equipos operativos. El área de Desarrollo es la de mayor tamaño, con un equipo de 11 trabajadores, seguida por el área Comercial con 8 trabajadores. Por su parte, el equipo de Customer Success está compuesto por 5 personas, y el de Marketing por 2. Es de notar que el departamento de Post-venta presenta una sub-estructura, en la cual el Gerente supervisa a una Jefa de Soporte, Daniela Contreras, quien lidera un equipo de 5 trabajadores.

%insertar imagen aqui
\insertimage[\label{img:organigrama}]{imagenes/organigrama.png}{scale=0.34}{Organigrama de Reservo, Agosto 2025. Fuente: elaboración propia}


% ------------------------------------------------------------------------------
% NUEVA SECCIÓN
% ------------------------------------------------------------------------------
\section{Descripción del problema u oportunidad}

\subsection{Antecedentes}
% Describir el problema que se enfrenta o la oportunidad de mejora que se desea introducir.

%Flujo General del Cliente por Área

\subsubsection{Proceso general}
La Figura \ref{img:proceso} muestra un diagrama del flujo general que sigue un cliente a lo largo de su ciclo de vida dentro de Reservo, visto desde la perspectiva de las áreas funcionales involucradas. El recorrido inicia con el primer contacto a través de Marketing, continúa con el proceso de calificación y venta por parte de los equipos SDR y Ejecutivos, y sigue con la etapa de incorporación (Onboarding) gestionada por Customer Success. Finalmente, el cliente interactúa con Soporte y entra en un ciclo de Retención, que culmina con su continuidad o su eventual pérdida (churn)

%insertar imagen aqui
\insertimage[\label{img:proceso}]{imagenes/proceso.png}{scale=0.65}{Flujo General del Cliente por Área. Fuente: elaboración propia}

%Estados del cliente / Ciclo de vida del cliente en reservo

\subsubsection{Etapa de Captación}

El proceso de captación, ilustrado en el anexo~\ref{img:captacion}, es iniciado por el área de \textbf{Marketing}, que ejecuta una acción específica para atraer el interés de potenciales clientes. Un \textbf{cliente} potencial encuentra esta acción de marketing y procede a registrar su información en la plataforma de Reservo. Este registro representa la primera transición crucial, cambiando el estado del \textbf{cliente} de un anónimo \textbf{`Desconocido'} a un \textbf{`Lead'} conocido dentro del sistema de Reservo.

Una vez enviados los datos, estos son recibidos por el equipo de \textbf{Ventas SDR} (Sales Development Representative), que realiza la tarea inicial de configurar el lead para su posterior contacto. El \textbf{SDR} procede a contactar al prospecto; en este punto, el flujo depende de la respuesta del \textbf{cliente}, ya que el proceso solo continúa si este contesta y acepta seguir adelante. Tras una interacción exitosa, el \textbf{SDR} ejecuta la tarea de `Calificar Lead' para determinar si el prospecto encaja con el perfil de cliente ideal de la empresa. Si la calificación es positiva, el \textbf{cliente} experimenta su segundo cambio de estado importante, convirtiéndose en un \textbf{`Lead calificado'}.

Para concluir esta fase, el equipo \textbf{SDR} se encarga de agendar una demostración con un \textbf{ejecutivo de ventas} y de enviar una cotización formal. La programación exitosa de esta demo marca la tercera y última transición de estado en este diagrama, elevando al prospecto a una \textbf{`Oportunidad de Venta'}. Con esto finaliza oficialmente el proceso de captación inicial y se transfiere al \textbf{cliente} a la siguiente etapa del embudo de ventas.

\subsubsection{Etapa de Ventas}

El proceso de Venta, detallado en el anexo~\ref{img:etapa_venta}, comienza cuando un \textbf{cliente} en estado de `Oportunidad de Venta' asiste a una demostración del producto. Una vez finalizada la demo, si el \textbf{cliente} decide contratar el servicio, el \textbf{Ejecutivo de Venta} procede a configurar al prospecto en el sistema, momento en el cual se produce la primera transición de estado importante, actualizando su condición a \textbf{`Precliente'}. A continuación, el \textbf{Ejecutivo de Venta} envía las instrucciones de pago y, una vez que el \textbf{cliente} completa dicha transacción, se realiza el cambio de estado final: el prospecto es convertido oficialmente en un \textbf{`Cliente'}. Finalmente, el \textbf{Ejecutivo de Venta} transfiere al nuevo \textbf{cliente} al área de \textbf{Customer Success}, dando inicio al `Proceso de Onboarding' y concluyendo así la etapa comercial.

\subsubsection{Etapa de Onboarding}

El proceso de Onboarding, como se detalla en el anexo~\ref{img:etapa_onboarding}, comienza inmediatamente después de que finaliza la venta, con el equipo de \textbf{Customer Success} preparando la cuenta e invitando al nuevo \textbf{cliente} a una primera capacitación. Una vez realizada esta sesión inicial, inicia una fase de seguimiento para monitorear si el \textbf{cliente} utiliza activamente la plataforma. Durante este período, ocurren las transiciones más relevantes, ya que, basándose en el nivel de riesgo detectado, un \textbf{cliente} puede ser clasificado como de \textbf{`riesgo alto'} o \textbf{`riesgo medio'}. Esta clasificación desencadena acciones de contacto proactivo por parte del equipo de \textbf{Customer Success}. Si el \textbf{cliente} no presenta riesgos y transcurre un período determinado, el proceso de Onboarding se da por finalizado, marcando la transición del \textbf{cliente} a un estado de uso regular de la herramienta.

\subsubsection{Etapa de Cliente Activo}

La etapa de \textbf{`Cliente Activo'}, ilustrada en el anexo~\ref{img:etapa_cliente_activo}, representa el ciclo de vida del \textbf{cliente} una vez finalizado el proceso de Onboarding, centrándose en la retención y expansión. En esta fase, el equipo de \textbf{Customer Success} realiza un monitoreo constante de la salud de la cuenta. Si se detecta un uso deficiente, se activan acciones de seguimiento, lo que implícitamente clasifica al \textbf{cliente} como un perfil \textbf{`en riesgo'} para mitigar una posible baja. Por el contrario, si la salud de la cuenta es buena, el equipo busca oportunidades de expansión (upselling); si el \textbf{cliente} acepta y completa el pago, su condición se actualiza, reafirmándolo como un \textbf{`cliente vigente'}. Paralelamente, el proceso gestiona las interacciones del \textbf{cliente} con el área de \textbf{Soporte} y los ciclos de facturación, donde un fallo recurrente en el pago puede culminar con la \textbf{pérdida del cliente (churn)}.

\subsubsection{Etapa de Cliente Inactivo}

El proceso de \textbf{`Cliente Inactivo'}, detallado en el anexo~\ref{img:etapa_cliente_inactivo}, gestiona tanto la baja de un \textbf{cliente} como su posible reactivación. El flujo se inicia cuando un \textbf{cliente} solicita cancelar su servicio, ya sea durante el Onboarding o en su ciclo de vida activo. El equipo de \textbf{Customer Success} procesa esta solicitud, formalizando la transición del \textbf{cliente} a un estado \textbf{`dado de baja'}. Posteriormente, se registran las razones de la pérdida y se informa al \textbf{cliente} para cerrar el ciclo. El diagrama también contempla un camino para la recuperación; si un \textbf{cliente} inactivo decide volver a contratar, el equipo de \textbf{Customer Success} gestiona el nuevo contrato y la reactivación de la cuenta. Esta acción revierte el estado del \textbf{cliente}, devolviéndolo a la condición de \textbf{`cliente activo'} y reintegrándolo al ciclo de vida principal.

\subsection{Problema encontrado}
Para asegurar su crecimiento en el competitivo mercado SaaS, Reservo debe optimizar continuamente el valor que extrae de su base de clientes. Actualmente, la organización enfrenta desafíos que limitan su capacidad para retener y expandir cuentas de manera eficiente. Estos desafíos se manifiestan en una serie de efectos observables que impactan directamente la rentabilidad y la sostenibilidad del negocio.

\subsubsection{Efectos y consecuencias observadas}
Los resultados de la gestión actual del ciclo de vida del cliente presentan tres consecuencias principales:

\begin{itemize}
    \item \textbf{Pérdida recurrente de clientes e ingresos:} La empresa experimenta una \textbf{Tasa de Churn} mensual de \textbf{2,46\,\%}, lo que se ha traducido en \textbf{70 bajas} en lo que va de 2025\footnote{La \textbf{Tasa de Churn} se define en la tabla~\ref{tab:estados-cliente}. El valor y el número de bajas corresponden a los datos internos de la empresa para el año 2025. Fuente: Base de datos de Reservo.}. Desde enero de 2022, el churn mensual promedia \textbf{2,77\,\%}; aunque la serie exhibe una \textit{tendencia} decreciente, el comportamiento histórico y reciente se aprecia en la Figura~\ref{img:churn3anios}.

    No obstante, el indicador acumula aproximadamente \textbf{dos años sin perforar el 2\,\%}, evidenciando un \textbf{estancamiento} en el rango 2–3\,\%. Esto es relevante porque los principales \textit{benchmarks} para empresas de suscripción (SaaS) recomiendan apuntar a \textbf{menos de 2\,\% de churn mensual} para situarse entre las mejores prácticas del mercado \citep{ChartMogul2023ChurnRate}. Mientras el churn permanezca por encima de ese umbral, la fuga de clientes seguirá impactando los ingresos recurrentes y condicionando el crecimiento neto.


    \item \textbf{Expansión y reactivación no sistemáticas:} Aunque la empresa logra generar valor adicional, los resultados no muestran un crecimiento acelerado. Durante 2024 se lograron \textbf{57 upsellings}, mientras que en 2025 se han conseguido \textbf{60} hasta la fecha. En cuanto a la reactivación, en 2024 se recuperaron \textbf{100 clientes}, mientras que en 2025 van \textbf{89}\footnote{Los datos de upsellings y clientes reactivados corresponden a los registros internos de la empresa. Fuente: Base de datos de Reservo.}. Estas cifras sugieren un crecimiento modesto en la expansión y una posible disminución en la capacidad de reactivar cuentas, indicando la falta de una estrategia proactiva y sistemática.

    \item \textbf{Asignación ineficiente de recursos de postventa:} Los esfuerzos de Customer Success exhiben \textit{baja efectividad} porque se interviene tarde y con ofertas poco ajustadas al perfil y necesidades del cliente en riesgo. No existe una clasificación operativa de riesgo durante el ciclo activo (más allá del onboarding), por lo que las acciones se activan cuando la probabilidad de abandono ya es alta; así, se prioriza reactivamente a clientes al borde del churn y se desaprovechan intervenciones tempranas con mayor probabilidad de retención y/o expansión.

\end{itemize}

%insertar imagen aqui
\insertimage[\label{img:churn3anios}]{imagenes/churn-3-anios.png}{scale=0.75}{Churn 3 años (ene-2022 a jul-2025) y tendencia. Fuente: elaboración propia}

\subsubsection{Análisis y Elección del Problema Central}
Los efectos descritos (pérdida de clientes, expansión no optimizada e ineficiencia) son síntomas de varios problemas subyacentes. Para seleccionar el problema central, se utiliza como base teórica el marco de \textbf{Gestión del Ciclo de Vida del Cliente (Customer Lifecycle Management)}, que postula que las acciones de una empresa deben adaptarse a la etapa y al comportamiento específico de cada cliente para ser efectivas.

Bajo esta óptica, la elevada tasa de abandono y el bajo crecimiento del valor no son problemas raíz, sino consecuencias directas de un problema más profundo y operativo. Se elige como problema central aquel que, de ser solucionado, tendría el mayor impacto en mitigar los demás. Por lo tanto, se descartan los síntomas y se selecciona la causa operativa fundamental que los origina. El problema central es:

\begin{center}
    \textit{Insuficiente caracterización y segmentación conductual de la cartera activa en postventa}
\end{center}

\subsubsection{Causas del Problema Central}
Este problema no se debe a una falta de datos, sino a la forma en que la empresa los interpreta y los utiliza para guiar sus acciones. Las causas fundamentales son:

\begin{itemize}
    \item \textbf{Toma de decisiones basada en indicadores de resultado:} Las estrategias se activan a partir de métricas retrospectivas, como la \textbf{`Tasa de Churn'} o un \textbf{`NPS'} de \textbf{52}\footnote{El \textbf{NPS} se define en la tabla~\ref{tab:estados-cliente}. El valor de 52 es el último resultado medido por la empresa. Fuente: Base de datos de Reservo.}. Estos indicadores miden un resultado que ya ocurrió, pero no permiten diferenciar el tratamiento entre clientes \textit{antes} de que el riesgo se materialice.

    \item \textbf{Análisis estático de la información:} La empresa analiza datos de manera estática, observando el estado actual de un cliente, pero no su evolución en el tiempo. Se carece de una visión longitudinal del comportamiento, lo que impide identificar tendencias, como una disminución gradual del uso de la plataforma, que son señales tempranas de un posible abandono.

    \item \textbf{Ausencia de criterios conductuales para la segmentación:} No se ha definido formalmente qué patrones de uso se correlacionan con un cliente saludable, en riesgo o con potencial de expansión. Sin estos criterios, es imposible segmentar la cartera de manera dinámica para aplicar acciones diferenciadas.

    \item \textbf{Enfoque operativo centrado en la reacción a eventos:} El flujo de trabajo del equipo de Customer Success está estructurado para reaccionar a eventos negativos (quejas, impagos) en lugar de estar diseñado para segmentar proactivamente la base de clientes y ejecutar acciones personalizadas.
\end{itemize}

\section{Descripción y Justificación del Proyecto}

La sección anterior definió como problema central la \textbf{insuficiente caracterización y segmentación conductual de la cartera activa en postventa}. A continuación, se presenta un proyecto diseñado para abordar directamente las causas de este problema, sentando las bases para una gestión del cliente más estratégica y proactiva basada en evidencia teórica y mejores prácticas gerenciales.

\subsection{Definición y Alcance del Proyecto}

El proyecto se acota explícitamente a la \textbf{etapa de Cliente Activo en postventa}, con foco en el \textbf{equipo de Customer Success}. El alcance contempla: (i) diseño y validación de un \textit{modelo analítico de segmentación conductual} de la cartera activa; (ii) definición de \textit{políticas de retención y expansión} por segmento; y (iii) lineamientos operativos para su aplicación por Customer Success. \textbf{No} incluye la implementación productiva en sistema ni desarrollos de software a medida: el entregable final es el \textit{diseño, la validación y la propuesta} del modelo y políticas asociadas. El estudiante desarrollará el modelo, documentará criterios y supuestos, construirá el datamart analítico y entregará guías de uso para el equipo.

\textit{Delimitación del estado de cliente (sólo “Cliente Activo”).} La decisión de trabajar únicamente con clientes activos responde a criterios de impacto, factibilidad y gobernabilidad. Impacto: la retención del cliente activo incide de forma directa en MRR y flujo de caja [\textit{espacio para insertar cifras internas}]. Factibilidad: este estado concentra datos de uso y eventos suficientes para segmentación conductual y \textit{scoring} mensual. Gobernabilidad: las palancas de intervención dependen del equipo de Customer Success y pueden ejecutarse sin cambios en adquisición, producto o pricing. El horizonte de \textbf{4 meses} exige foco en una fase con datos disponibles y capacidad de acción inmediata.

\subsection{Justificación Estratégica}

La realización de este proyecto se justifica por su alineación directa con la necesidad de la empresa de optimizar la rentabilidad de su cartera y asegurar un crecimiento sostenible. La gestión proactiva del cliente basada en datos es un eslabón crítico para la competitividad y la rentabilidad en mercados digitales \citep{Kumar2010}.

Las razones estratégicas se basan en tres pilares principales:  
\begin{itemize}
    \item \textbf{Mitigar la pérdida de ingresos por churn:} La tasa de churn es un indicador central en modelos de suscripción. Reducirla suele ser más rentable que adquirir nuevos clientes \citep{Gupta2006}. La tasa de abandono mensual de \textbf{2.46\,\%} afecta los ingresos recurrentes. La serie histórica se muestra en la Figura~\ref{img:churn3anios}. Un modelo predictivo permitiría anticipar señales de deserción y activar acciones de retención personalizadas, en línea con la teoría del ciclo de vida del cliente \citep{Lemmens2008}.
    \item \textbf{Capitalizar oportunidades de expansión:} El aumento del valor de vida del cliente (CLV) mediante \textit{upselling} y \textit{cross-selling} segmentado permite sistematizar la expansión \citep{Kumar2010}. La evolución observada (57 en 2024 frente a 60 en 2025) sugiere potencial por capturar.
    \item \textbf{Optimizar el esfuerzo de Customer Success:} El trato homogéneo de la cartera genera sobrecostos y baja efectividad. La segmentación basada en datos permite focalizar recursos en clientes de mayor riesgo o mayor potencial y se alinea con prácticas de CRM estratégico \citep{Kumar2010}.
\end{itemize}

\subsection{Evaluación Económica}

La evaluación estimará el impacto incremental en ingresos y margen. Se calculará el MRR retenido incremental por reducción de churn y el MRR incremental por expansión, descontando costos directos del proyecto. Este marco sigue el enfoque de CRM orientado a valor económico del cliente y uso de CLV para guiar decisiones \citep{Kumar2010}:

\[
\Delta \mathrm{MRR}_{\mathrm{retenido}} = \mathrm{MRR}_{\mathrm{base}} \times (\mathrm{Churn}_{\mathrm{base}} - \mathrm{Churn}_{\mathrm{post}})
\]
\[
\Delta \mathrm{MRR}_{\mathrm{exp}} = \mathrm{MRR}_{\mathrm{base}} \times (\mathrm{TasaExp}_{\mathrm{post}} - \mathrm{TasaExp}_{\mathrm{base}})
\]
\[
\mathrm{Beneficio\ anual\ esperado} \approx (\Delta \mathrm{MRR}_{\mathrm{retenido}} + \Delta \mathrm{MRR}_{\mathrm{exp}}) \times \mathrm{Margen} \times 12 - \mathrm{Costo\ del\ proyecto}
\]

\textbf{Dimensionamiento de la base y piso económico.} Se establecerá un \textbf{piso económico} para comparar resultados al cierre del proyecto. Se cuantificarán \(N_{\mathrm{activos}}\), \(\mathrm{MRR}_{\mathrm{base}}\), \(\mathrm{ARPA} = \mathrm{MRR}_{\mathrm{base}}/N_{\mathrm{activos}}\), \(\mathrm{Churn}_{\mathrm{base}}\) y Margen. El piso exigirá que el beneficio anual esperado cubra el costo del proyecto y logre \textbf{payback} en el horizonte definido [\textit{espacio para metas internas}]. Se reportará el \textbf{MRR perdido por churn} como
\[
\mathrm{MRR}_{\mathrm{perdido}} = \mathrm{MRR}_{\mathrm{base}} \times \mathrm{Churn}_{\mathrm{base}}.
\]

\textbf{Valor de cliente (CLV) y costo de una baja.} Se estimará el \textbf{CLV} bajo supuestos de tasa de churn mensual constante y margen bruto, siguiendo formulaciones estándar de la literatura de CLV \citep{Gupta2006}:
\[
\mathrm{CLV}_{\mathrm{simple}} = \mathrm{ARPA} \times \mathrm{Margen} \times \frac{1}{\mathrm{Churn}_{\mathrm{base}}}
\]
\[
\mathrm{CLV}_{\mathrm{descontado}} = \frac{\mathrm{ARPA} \times \mathrm{Margen}}{\mathrm{Churn}_{\mathrm{base}} + i}
\]
donde \(i\) es la tasa de descuento mensual. La \textbf{pérdida económica por baja} se aproximará al CLV remanente del cliente y servirá para dimensionar el impacto de una reducción marginal del churn. \textbf{[Espacio para ARPA, margen e \(i\)].}

\subsection{Indicadores de Éxito (uno por objetivo específico)}

\begin{itemize}
    \item \textbf{Objetivo 1 — Conjunto validado de variables de comportamiento:} AUC del modelo baseline construido con dichas variables \(\geq 0{,}70\) en test temporal. Este umbral se considera aceptable para discriminar eventos frente a no eventos en modelos de clasificación binaria \citep{Hosmer2013}.
    \item \textbf{Objetivo 2 — Datamart estructurado:} Cobertura del snapshot mensual \(\geq 90\%\) de la cartera activa (clientes con todas las variables requeridas disponibles en la ventana de cierre). Este criterio asegura suficiencia de datos para la operación de CRM analítico y reduce sesgos por datos faltantes \citep{Kumar2010}.
    \item \textbf{Objetivo 3 — Segmentos definidos y caracterizados:} \textbf{Porcentaje de churn capturado en el 20\% superior del ranking de riesgo} \(\geq 40\%\) en test temporal; esto equivale a que el 20\% de cuentas con mayor score concentra al menos el 40\% de los abandonos esperados, conforme a curvas de \textit{gains} acumulados utilizadas en marketing analítico \citep{SAS41683}.
    \item \textbf{Objetivo 4 — Portafolio de políticas por segmento:} Tasa de aceptación de ofertas o acciones en clientes de alto riesgo \(\geq 25\%\) durante el piloto inicial. Meta operativa coherente con principios de segmentación y personalización del CRM \citep{Kumar2010}.
\end{itemize}
Las métricas se medirán al cierre del proyecto (mes 4) y se realizará un seguimiento de verificación a 90 días posteriores. Criterio de aceptación: el proyecto se considerará exitoso si se cumplen al menos 3 de los 4 indicadores y se entrega el datamart y el portafolio de políticas documentado.


\subsection{Razones Prácticas y Técnicas}

El enfoque propuesto considera tiempo disponible de aproximadamente 4 meses, recursos existentes en la empresa (bases de datos, infraestructura para analítica y BI) y capacidad del estudiante para construir un flujo reproducible de trabajo. La segmentación conductual permite resultados accionables en el corto plazo y se alinea con prácticas de CRM analítico y gestión del ciclo de vida del cliente \citep{Kumar2010}. El estudiante priorizará entregables con trazabilidad metodológica y guía operativa para el equipo de Customer Success.

\subsection{Aspectos Fuera del Alcance}

Quedan fuera del proyecto: (i) etapas distintas de “Cliente Activo” (captación, evaluación/trial, onboarding, moroso/inactivo); (ii) campañas de marketing y su creatividad; (iii) implementación productiva en tiempo real o integraciones de software; (iv) cambios de pricing o facturación; (v) definición de gobierno de datos y seguridad más allá de lo necesario para el análisis; (vi) experimentación A/B a gran escala. El estudiante dejará pautas metodológicas y criterios para una futura ejecución por las áreas correspondientes.

\subsection{Elección del Enfoque}

Se elige un \textbf{modelo analítico de segmentación conductual} con \textit{scoring} y reglas de acción por sobre alternativas comunes. Reportes descriptivos generales no entregan prioridad operativa. Reglas heurísticas estáticas presentan menor poder predictivo y adaptabilidad. Despliegues de automatización o ML en tiempo real no son viables en el horizonte definido. El enfoque propuesto balancea impacto, interpretabilidad y factibilidad con los recursos disponibles, y se alinea con la literatura de CRM y gestión de clientes \citep{Kumar2010}.

\section{Objetivos del proyecto}

\subsection{Objetivo general}

Desarrollar un modelo analítico de segmentación conductual para la cartera de clientes de Reservo, que permita diferenciar y gestionar de manera proactiva las estrategias de retención y expansión para cada tipo de cliente.

\subsection{Objetivos específicos}

Para alcanzar el objetivo general, se deben generar los siguientes resultados específicos, cada uno de los cuales aborda una de las causas del problema central:

\begin{itemize}
    \item Obtener un conjunto validado de variables de comportamiento que caractericen a los clientes y permitan predecir el riesgo de abandono y las oportunidades de expansión.
    
    \item Construir un repositorio estructurado de datos (datamart) que consolide la información de clientes y habilite el análisis de comportamiento y el desarrollo de modelos predictivos.
    
    \item Definir segmentos de clientes definidos y caracterizados en función de sus patrones de comportamiento dinámicos y su evolución en el tiempo.
    
    \item Diseñar un portafolio de políticas de negocio personalizadas para cada segmento identificado, orientadas a guiar las acciones de los equipos de Marketing, Ventas y Customer Success.

    \item Implementar un prototipo operativo del modelo analítico y del portafolio de políticas en la empresa, con criterios de éxito y lineamientos de adopción.

\end{itemize}

\section{Marco conceptual}

\section{Metodología}

\section{Desarrollo y Resultados}

\section{Cronograma}
El plan de trabajo del proyecto se ha estructurado en un horizonte temporal de 18 semanas, abarcando el período comprendido entre agosto y diciembre de 2025. Para asegurar una ejecución metódica y el cumplimiento de los objetivos, el proyecto se ha dividido en cuatro fases secuenciales, cada una con entregables y actividades específicas.

\subsection{Actividades principales} %Ver si se puede agregar cosas del CRISP DM
La división del proyecto en fases permite una gestión estructurada del alcance y facilita el seguimiento de los avances. A continuación, se describen las actividades fundamentales de cada etapa:

\begin{itemize}
    \item \textbf{Fase 1: Planificación y Fundamentos.} Esta etapa inicial establece las bases conceptuales y operativas del proyecto. Comprende el levantamiento de requerimientos con los \textit{stakeholders}, la formalización del árbol de problemas, y la definición del alcance y los objetivos. Esta fase culmina con la validación de la línea base sobre la cual se medirá el éxito del proyecto.

    \item \textbf{Fase 2: Análisis de Datos y Desarrollo del Modelo.} Corresponde al núcleo analítico del trabajo. Se inicia con la recopilación, limpieza y pre-procesamiento de los datos históricos de la empresa. Posteriormente, se ejecuta un análisis exploratorio para identificar variables de comportamiento relevantes y, finalmente, se procede con el desarrollo y la validación iterativa del modelo de segmentación.

    \item \textbf{Fase 3: Diseño de la Propuesta Estratégica.} En esta fase, los resultados del modelo analítico se traducen en valor de negocio. Se realiza una caracterización detallada de cada segmento de cliente identificado y se diseña un portafolio de políticas de retención y expansión personalizadas. El trabajo concluye con la consolidación de todos los hallazgos en el informe final.

    \item \textbf{Fase 4: Cierre y Presentación Final.} La última etapa se dedica a la comunicación de los resultados. Se prepara el material de apoyo y se realiza la presentación final del proyecto, sus conclusiones y recomendaciones a la organización, junto con la entrega de toda la documentación y artefactos generados.
\end{itemize}

\subsection{Tiempos estimados}
La siguiente tabla presenta la distribución de las actividades en el tiempo, detallando las tareas específicas de cada fase y su duración estimada en semanas.

\begin{table}[H]
  \begin{threeparttable}
    \centering
    \scriptsize
    \caption{Cronograma Detallado de Fases y Actividades del Proyecto}
    \renewcommand{\arraystretch}{1.5}
    \begin{tabular}{|l|p{9cm}|c|}
      \hline
      \textbf{Fase} & \textbf{Actividades Detalladas} & \textbf{Duración (Semanas)} \\
      \hline
      \textbf{1. Planificación} & 
      - Levantamiento y validación de requerimientos. \newline
      - Definición del problema, objetivos y alcance. \newline
      - Elaboración del primer informe de avance. & 
      \textbf{4} \\
      \hline
      \textbf{2. Desarrollo del Modelo} & 
      - Recopilación y pre-procesamiento de datos. \newline
      - Análisis exploratorio de datos (EDA). \newline
      - Desarrollo y entrenamiento del modelo. \newline
      - Validación y ajuste de métricas de rendimiento. & 
      \textbf{9} \\
      \hline
      \textbf{3. Propuesta Final} & 
      - Caracterización de segmentos de clientes. \newline
      - Diseño de políticas de negocio personalizadas. \newline
      - Redacción del informe final. & 
      \textbf{4} \\
      \hline
      \textbf{4. Cierre} & 
      - Preparación del material de apoyo. \newline
      - Presentación final de resultados. & 
      \textbf{1} \\
      \hline
      \multicolumn{2}{|r|}{\textbf{Duración Total Estimada}} & \textbf{18} \\
      \hline
    \end{tabular}
    \label{tab:cronograma}
  \end{threeparttable}
\end{table}

\newpage

\section{}
% ------------------------------------------------------------------------------
% REFERENCIAS, revisar configuración \stylecitereferences \def\stylecitereferences {natbib} 
% ------------------------------------------------------------------------------
\clearpage
\bibliographystyle{apalike}
\bibliography{library}
% Listar fuentes usadas en Antecedentes y Justificación, y referencias científicas del informe.
% Usar formato APA 7 para citas y bibliografía.


% ------------------------------------------------------------------------------
% ANEXO
% ------------------------------------------------------------------------------
\newpage

\begin{appendixd}

	\section{Estados del cliente}

    \subsection{}
    \insertimage[\label{img:captacion}]{bizagi/captacion.png}{scale=0.27}{Etapa de Adquisición en Reservo. Fuente: elaboración propia}
    \subsection{}
    \insertimage[\label{img:etapa_venta}]{bizagi/venta.png}{scale=0.25}{Etapa de Venta en Reservo. Fuente: elaboración propia}
    \subsection{}
    \insertimage[\label{img:etapa_onboarding}]{bizagi/onboarding.png}{scale=0.19}{Etapa de Onboarding del cliente en Reservo. Fuente: elaboración propia}
    \subsection{}
    \insertimage[\label{img:etapa_cliente_activo}]{bizagi/cliente_activo.png}{scale=0.15}{Etapa de cliente activo en Reservo. Fuente: elaboración propia}
    \subsection{}
    \insertimage[\label{img:etapa_cliente_inactivo}]{bizagi/cliente_inactivo.png}{scale=0.30}{Etapa de Cliente inactivo en Reservo. Fuente: elaboración propia}

    	% Imagen, se numerará automáticamente con la letra del anexo según
	% la configuración \appendixindepobjnum
	  \insertimage[\label{img:proceso_completo}]{bizagi/proceso_completo.png}{scale=0.05}{Proceso completo de cliente en Reservo. Fuente: elaboración propia}

%Tabla con indicadores

  \section{Indicadores del proceso}
\begin{table}[H]
  \begin{threeparttable}
    \centering
    \scriptsize
    \caption{Métricas asociadas a clientes. Fuente: Elaboración propia}
    \renewcommand{\arraystretch}{1.3}
    \begin{tabular}{|l|C{3.5cm}|C{3.3cm}|C{5.5cm}|}
      \hline
      \textbf{Estado} & \textbf{Definición} & \textbf{Métricas asociadas} & \textbf{Fórmula / Cálculo} \\
      \hline
      Desconocido & Persona que aún no ha interactuado con la marca. & Visitas al sitio web. &
      Visitas web: nº de sesiones únicas registradas en la web. \\
      \hline
      Lead & Persona que mostró interés inicial y dejó sus datos de contacto. &
      Costo por Lead (CPL). &
      CPL = $\tfrac{\text{Gasto en marketing}}{\text{Nº leads generados}}$ \\
      \hline
      Lead calificado & Lead que ha sido validado por el equipo SDR como un cliente potencial. &
      Tasa de conversión a SQL (Sales Qualified Lead). &
      SQL rate = $\tfrac{\text{Nº leads calificados}}{\text{Nº leads totales}} \times 100$ \\
      \hline
      Oportunidad de venta & Lead calificado que ha aceptado una demostración o propuesta formal. &
      Win Rate. &
      Win rate = $\tfrac{\text{Nº ventas cerradas}}{\text{Nº oportunidades}} \times 100$ \\
      \hline
      Cliente nuevo & Ha contratado el servicio y se ha completado su alta en el sistema. &
      Costo de Adquisición de Cliente (CAC). &
      CAC = $\tfrac{\text{Costo total de Marketing y Ventas}}{\text{Nº clientes nuevos adquiridos}}$ \\
      \hline
      Onboarding (0--90 días) & Cliente nuevo en el proceso de implementación y capacitación inicial. &
      Tasa de Activación. &
      Activación = $\tfrac{\text{Nº clientes que completan onboarding}}{\text{Nº clientes nuevos}} \times 100$ \\
      \hline
      Cliente activo & Cliente que ha completado el onboarding y usa la plataforma regularmente. &
      Engagement Score. &
      Engagement = Frecuencia y profundidad de uso de funciones clave. \\
      \hline
      Cliente en riesgo & Cliente con bajo uso, quejas recurrentes o problemas de pago. &
      \% de Clientes en Riesgo. &
      \% en riesgo = $\tfrac{\text{Nº clientes en riesgo}}{\text{Nº clientes activos}} \times 100$ \\
      \hline
      Cliente en expansión & Cliente activo que contrata nuevas funcionalidades, módulos o sedes. &
      Net Dollar Retention (NDR). &
      NDR = $\tfrac{\text{MRR inicial + Expansión - Churn}}{\text{MRR inicial}} \times 100$ \\
      \hline
      Cliente Churn & Cliente que ha cancelado su suscripción y no renueva. &
      Tasa de Churn. &
      Churn Rate = $\tfrac{\text{Nº clientes perdidos}}{\text{Nº clientes al inicio del período}} \times 100$ \\
      \hline
      Valor de Vida del Cliente & Predicción del beneficio neto atribuido a toda la relación futura con un cliente. &
      Lifetime Value (LTV). &
      LTV = (Ticket promedio $\times$ Recurrencia) $\times$ Vida del cliente \\
      \hline
    \end{tabular}
    \label{tab:estados-cliente}
  \end{threeparttable}
\end{table}

\end{appendixd}

% --- Fin cuerpo_informe.tex ---


