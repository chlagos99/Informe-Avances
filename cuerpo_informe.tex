% Template:     Informe LaTeX
% Documento:    Archivo de ejemplo
% Versión:      8.3.8 (07/02/2025)
% Codificación: UTF-8
%
% Autor: Pablo Pizarro R.
%        pablo@ppizarror.com
%
% Manual template: [https://latex.ppizarror.com/informe]
% Licencia MIT:    [https://opensource.org/licenses/MIT]


% ------------------------------------------------------------------------------
% NUEVA SECCIÓN
% ------------------------------------------------------------------------------

% --- Inicio cuerpo_informe.tex ---

\section{Antecedentes generales}
\subsection{Industria y ambiente competitivo}


El sector de la salud en Chile y en toda Latinoamérica está experimentando una transformación digital irreversible. Este mercado, conocido como HealthTech o salud digital, alcanzó una valoración de \textbf{\$17 mil millones de dólares en 2024} en toda la región y se proyecta que su crecimiento se acelere. Las estimaciones indican que el mercado podría alcanzar los \textbf{\$58 mil millones de dólares para 2030}, lo que representa una \textbf{tasa de crecimiento anual compuesta (CAGR) del 23.2\%} durante este periodo\footnote{Grand View Research. ``Latin America Digital Health Market Size \& Outlook, 2030''.}. Este dinamismo no es casual, sino que es impulsado por la creciente adopción de tecnologías digitales, el aumento de la demanda de telemedicina y la necesidad de mejorar la eficiencia en la atención sanitaria\footnote{Mordor Intelligence. ``Latin America Digital Transformation Market Size \& Share Analysis - Growth Trends \& Forecasts (2025 - 2030)''.}.

El principal catalizador de este cambio en Latinoamérica y Chile ha sido la aceleración digital impulsada por la pandemia, que expandió rápidamente la telemedicina y sentó las bases de un modelo híbrido de atención.\footnote{Superintendencia de Salud de Chile. ``Telemedicina y Salud Mental en Chile'' (Documento de trabajo, 2025). Ley N.\,21.541 formaliza directrices para la práctica, seguridad de datos e interoperabilidad.} El impulso post COVID-19 redefinió expectativas: pacientes y equipos valoran la continuidad de cuidados, el acceso a especialistas a distancia y la coordinación entre niveles, pero también se observan asimetrías de recursos y brechas de adopción que condicionan el uso sostenido.\footnote{Revista Chilena de Salud Pública (U. de Chile). ``Telemedicina en Chile: uso, desarrollo y controversias en una red de salud pública del sur de Chile'' (estudio cualitativo).} En atención primaria, la estrategia de telesalud se ha usado para gestionar demanda y acceso, apoyando la resolución oportuna y la articulación con especialidades.\footnote{Pan American Journal of Public Health. ``Telesalud: una estrategia digital para la gestión de la demanda en APS en Chile'' (2024).} Con este contexto, para prestadores y clínicas en Chile, la adopción efectiva de tecnologías —desde teleconsulta hasta autogestión de citas y canales asincrónicos— deja de ser ventaja y se vuelve condición para mantener relevancia y calidad asistencial.\footnote{Superintendencia de Salud de Chile. ``Telemedicina y Salud Mental en Chile'' (2025).}


Este entorno de alta demanda ha generado un ecosistema de proveedores de software sumamente competitivo y diverso. Se pueden identificar tres perfiles principales de actores en el mercado:

\begin{itemize}
    \item \textbf{Startups y Empresas Nativas Digitales:} Son compañías ágiles, altamente especializadas en nichos específicos y con una gran capacidad de innovación. Sin embargo, a menudo enfrentan desafíos para escalar y ganar la confianza del mercado.
    
    \item \textbf{Empresas Locales Consolidadas:} Poseen un profundo conocimiento del mercado chileno, incluyendo sus complejidades regulatorias y culturales. Suelen tener una base de clientes estable, aunque pueden ser más lentas en la adopción de nuevas tecnologías.
    
    \item \textbf{Actores Internacionales:} Cuentan con enormes recursos, tecnología avanzada y marcas reconocidas globalmente. Su principal desafío es adaptar sus soluciones estandarizadas a las particularidades del sistema de salud y las normativas chilenas, como la Ley de Derechos y Deberes del Paciente\footnote{Ley N° 20.584, que regula los derechos y deberes que tienen las personas en relación con acciones vinculadas a su atención en salud. Fuente: Biblioteca del Congreso Nacional de Chile. \url{https://www.bcn.cl/leychile/navegar?idNorma=1039348}}, que impone estrictos requisitos sobre la seguridad y privacidad de la ficha clínica.
\end{itemize}

En este escenario, el éxito no depende únicamente de la tecnología, sino de la capacidad de una empresa para navegar estos desafíos. La diferenciación se logra a través de la hiper-especialización (ofrecer una solución perfecta para un tipo de especialista), la usabilidad (crear plataformas intuitivas que requieran una mínima capacitación) y, fundamentalmente, la calidad del soporte al cliente. Las empresas que logran combinar estos elementos son las que consiguen posicionarse sólidamente, convirtiendo el software de una simple herramienta a un socio estratégico para el crecimiento del profesional de la salud.

\subsection{Características de la empresa u organización}

RESERVO es una empresa chilena de tecnología SaaS\footnote{Software as a Service} cuyo rol en la industria es ser un \textbf{socio tecnológico integral para profesionales y centros de salud de tamaño pequeño y mediano}\footnote{Información obtenida del sitio web oficial de la empresa, sección "Quiénes Somos". Consultado en agosto de 2025. \url{https://www.reservo.cl/quienes-somos}}. Su modelo de negocio se centra en un nicho de mercado que históricamente ha sido desatendido por las grandes corporaciones de software, las cuales se enfocan en hospitales y grandes cadenas de clínicas. RESERVO atiende específicamente a especialistas independientes, dentistas, kinesiólogos, psicólogos y clínicas medianas, describiendo así las características de la organización.

La oportunidad que la empresa busca introducir se basa en un problema claro que enfrentan sus potenciales clientes: estos profesionales a menudo invierten una cantidad desproporcionada de su tiempo en tareas administrativas (gestión de agendas, recordatorios de citas, cobros) en lugar de en la atención al paciente. Además, muchos carecen de las herramientas y el conocimiento para competir digitalmente y atraer nuevos pacientes, lo que representa una oportunidad de mejora\footnote{Análisis basado en una entrevista con el Gerente de Producto de RESERVO, realizada el 15 de agosto de 2025.}.

Para resolver estos problemas, el producto principal de RESERVO es una \textbf{plataforma centralizada y modular} que ofrece una solución completa\footnote{Descripción de funcionalidades basada en la información pública disponible en la sección "Características" del sitio web oficial de RESERVO. \url{https://www.reservo.cl/caracteristicas}}:

\begin{itemize}
    \item \textbf{Eficiencia Operativa:} Automatiza la gestión de la agenda, el envío de recordatorios por WhatsApp y correo electrónico, y centraliza las fichas clínicas de los pacientes en un solo lugar.
    
    \item \textbf{Gestión Financiera:} Simplifica la emisión de boletas y el control de ingresos, proporcionando reportes para una mejor toma de decisiones.
    
    \item \textbf{Marketing y Presencia Digital:} Ofrece herramientas para la creación de un sitio web profesional con un sistema de reservas en línea integrado, permitiendo a los profesionales captar pacientes 24/7.
\end{itemize}

Este enfoque no solo entrega un software, sino que ofrece una \textbf{solución accesible y escalable} que se adapta a las necesidades particulares de sus usuarios. De esta forma, RESERVO se posiciona como un catalizador que permite a sus clientes optimizar su eficiencia, aumentar su rentabilidad y consolidar su presencia digital en un mercado altamente competitivo, definiendo así su rol dentro de la industria.

La estructura organizacional de la empresa, detallada en el Organigrama General (Figura \ref{img:organigrama}), se organiza de manera jerárquica y funcional bajo la dirección del CEO, Pablo Saintard. Reportando directamente a él se encuentran cinco gerencias o liderazgos que definen las áreas clave de la compañía: Post-venta, Customer Success, Comercial, Marketing y Desarrollo. Cada una de estas áreas está a cargo de un responsable directo: Sebastián Cruzat (Gerente Postventa), Natalia Contreras (Customer Success Manager), Sebastian Piñeiro (Gestor de Ventas), Diana Álvarez (Marketing Leader) y Sebastián Concha (Gerente TI).

A su vez, cada líder supervisa a sus respectivos equipos operativos. El área de Desarrollo es la de mayor tamaño, con un equipo de 11 trabajadores, seguida por el área Comercial con 8 trabajadores. Por su parte, el equipo de Customer Success está compuesto por 5 personas, y el de Marketing por 2. Es de notar que el departamento de Post-venta presenta una sub-estructura, en la cual el Gerente supervisa a una Jefa de Soporte, Daniela Contreras, quien lidera un equipo de 5 trabajadores.

%insertar imagen aqui
\insertimage[\label{img:organigrama}]{imagenes/organigrama.png}{scale=0.34}{Organigrama de Reservo, Agosto 2025}

\section{Descripción del problema u oportunidad}

\subsection{Problema u oportunidad identificada}
% Describir el problema que se enfrenta o la oportunidad de mejora que se desea introducir.

%Flujo General del Cliente por Área

La Figura \ref{img:proceso} muestra un diagrama del flujo general que sigue un cliente a lo largo de su ciclo de vida dentro de Reservo, visto desde la perspectiva de las áreas funcionales involucradas. El recorrido inicia con el primer contacto a través de Marketing, continúa con el proceso de calificación y venta por parte de los equipos SDR y Ejecutivos, y sigue con la etapa de incorporación (Onboarding) gestionada por Customer Success. Finalmente, el cliente interactúa con Soporte y entra en un ciclo de Retención, que culmina con su continuidad o su eventual pérdida (churn)

%insertar imagen aqui
\insertimage[\label{img:proceso}]{imagenes/proceso.png}{scale=0.65}{Flujo General del Cliente por Área}

%Estados del cliente / Ciclo de vida del cliente en reservo
En el anexo \ref{img:etapa_adquisicion} se tiene la figura que muestra el comienzo del ciclo de vida del cliente y potenciales clientes en Reservo. El potencial cliente encuentra Reservo gracias al equipo de marketing, pasando de un estado inicial \textbf{desconocido} a través de una \textit{interacción con marketing} que lo convierte en un \textbf{lead}. Este proceso marca la transición del área de \textit{Marketing} al área de \textit{Ventas SDR}, donde el lead es sometido a un proceso de \textit{calificación SDR} para evaluar su ajuste, interés y potencial de compra. Los leads que superan exitosamente esta calificación se convierten en \textbf{leads calificados}, los cuales son transferidos a la \textbf{Etapa de Venta} para continuar con el proceso de cierre comercial.

En el anexo \ref{img:etapa_venta} se presenta la \textbf{Etapa de Venta}, donde los leads calificados provenientes de la etapa anterior son gestionados por los \textit{Ejecutivos de Venta}. El proceso inicia con una \textbf{demo + propuesta} que genera una \textbf{oportunidad de venta}. Posteriormente, se evalúa si \textbf{¿se cierra el contrato?}: en caso afirmativo (\textit{Sí}), el lead se convierte en un \textbf{contrato cerrado} y avanza a la \textbf{Etapa de Onboarding}; en caso negativo (\textit{No}), el lead retorna al estado de \textbf{lead calificado} para futuras oportunidades de conversión.

En el anexo \ref{img:etapa_onboarding} se muestra la \textbf{Etapa de Onboarding}, gestionada por el equipo de \textit{Customer Success}. El proceso comienza con un \textbf{cliente nuevo} que procede al \textbf{alta y primera configuración} del sistema. Posteriormente, el cliente ingresa al período de \textbf{onboarding (0-90 días)}, durante el cual recibe acompañamiento y capacitación para la adopción efectiva de la plataforma. Una vez completado este período, el cliente transita al \textbf{Ciclo de Vida del Cliente Activo}, donde continúa su relación con Reservo como usuario establecido.

En el anexo \ref{img:etapa_cliente_activo} se presenta el \textbf{Ciclo de Vida del Cliente Activo}, gestionado por los equipos de \textit{Customer Success} y \textit{Soporte}. El proceso se centra en un \textbf{cliente activo} que puede seguir múltiples trayectorias: mediante estrategias de \textit{upsell/cross-sell} puede convertirse en \textbf{cliente en expansión} y retornar al ciclo activo; a través de procesos de \textit{renovación} transita a \textbf{cliente retenido}; o mediante \textit{alertas} por bajo uso, quejas o impago puede pasar a \textbf{cliente en riesgo}. Desde el estado de riesgo, se evalúa \textbf{¿se recupera?}: si la respuesta es afirmativa (\textit{Sí}), regresa a cliente retenido; si es negativa (\textit{No}), se convierte en \textbf{cliente churn (pérdida)}. Finalmente, se determina \textbf{¿se reactiva?}: en caso positivo (\textit{Sí}), el cliente retorna al ciclo activo; en caso negativo (\textit{No}), permanece como cliente perdido.

\subsection{Problema identificado}

En la actualidad, RESERVO se encuentra en una posición crítica dentro del competitivo mercado SaaS de salud en Chile y Latinoamérica. A pesar de contar con más de 3 500 clientes activos —entre especialistas independientes, clínicas medianas y otros profesionales de la salud—, \textbf{la empresa no aprovecha de forma profunda la información de uso y comportamiento disponible}, limitándose a indicadores superficiales que impiden identificar con antelación señales de riesgo (bajo uso, quejas o impago) y oportunidades de expansión (upsell/cross-sell). Esta \textbf{falta de análisis detallado y segmentación avanzada} fuerza al equipo de Customer Success a actuar de manera reactiva, provocando la pérdida de suscripciones retenibles, el desaprovechamiento de oportunidades de aumento de ingresos por cliente y la ineficiente asignación de recursos operativos.


\subsection{Causas Fundamentales del Problema}

\subsubsection{Causas Primarias}

\textbf{Ausencia de un modelo integral de análisis comportamental}: La empresa carece de un sistema estructurado para analizar y segmentar a sus clientes basándose en patrones de uso, adopción de funcionalidades y comportamiento a lo largo del ciclo de vida. Esta deficiencia se manifiesta en múltiples niveles: no existe claridad sobre qué características poseen los clientes que realizan upgrades exitosos, no se identifican patrones de subutilización de funcionalidades pagadas (como clientes con planes de boletas que no las integran), y no se detectan tempranamente las señales de deterioro en la actividad que preceden al churn.

\textbf{Desconexión entre datos disponibles y decisiones estratégicas}: A pesar de contar con información valiosa sobre el comportamiento de los clientes, RESERVO no ha logrado traducir estos datos en políticas de marketing diferenciadas y acciones proactivas. La empresa identifica que tiene clientes que facturan 100 millones de pesos pero pagan solo 40 mil pesos de mensualidad, evidenciando una desalineación entre el valor percibido por el cliente y el precio cobrado, pero no cuenta con los mecanismos para abordar sistemáticamente estas asimetrías.

\textbf{Enfoque reactivo en la gestión de clientes}: El área de Customer Success opera principalmente de manera reactiva, respondiendo a problemas cuando ya se han manifestado (bloqueos, quejas, cancelaciones) en lugar de anticiparlos mediante el análisis predictivo de comportamientos. Esta aproximación resulta en una mayor tasa de churn evitable y en la pérdida de oportunidades de expansión que podrían haberse identificado mediante el análisis proactivo de patrones de uso.

\subsubsection{Causas Secundarias}

\textbf{Segmentación limitada y genérica}: La empresa maneja a sus 3{,}500 clientes con estrategias de marketing y retención relativamente homogéneas, sin considerar las diferencias significativas en comportamiento, necesidades y potencial de crecimiento entre distintos segmentos. Esta aproximación ``talla única'' resulta ineficiente en un mercado SaaS donde la personalización y el enfoque diferenciado son factores críticos de éxito.

\textbf{Visibilidad fragmentada del cliente}: La falta de unificación de estadísticas para clientes grandes con múltiples sedes impide una comprensión holística del valor y comportamiento de estos accounts de alto valor. Esta fragmentación dificulta la toma de decisiones estratégicas sobre pricing, soporte especializado y estrategias de expansión.

\subsection{Contexto Específico en el Ciclo de Vida del Cliente}

El problema se manifiesta de manera diferente en cada etapa del ciclo de vida del cliente desarrollado por RESERVO:

\textbf{En la Etapa de Onboarding (0-90 días)}: La falta de segmentación comportamental impide personalizar el proceso de onboarding según el perfil y necesidades específicas del cliente, resultando en tasas de adopción subóptimas y un mayor tiempo hasta el ``first value''.

\textbf{En el Ciclo de Vida del Cliente Activo}: La ausencia de un modelo predictivo impide la identificación temprana de clientes en riesgo de churn, la detección oportuna de oportunidades de expansión, y la optimización de las estrategias de retención según el comportamiento específico de cada segmento.

\textbf{En los procesos de Renovación}: Sin insights sobre los patrones de uso que correlacionan con alta satisfacción y propensión a renovar, el equipo de Customer Success carece de argumentos basados en datos para las conversaciones de renovación y no puede anticipar adecuadamente qué clientes requieren atención especial durante este proceso crítico.

Esta problemática multifacética requiere una solución integral que combine análisis avanzado de datos, segmentación comportamental, desarrollo de políticas de marketing diferenciadas, y la implementación de procesos automatizados que permitan escalar las intervenciones personalizadas a través de toda la base de clientes de RESERVO.

\subsection{Consecuencias del Problema}

\subsubsection{Impactos Financieros Directos}

\textbf{Pérdida de ingresos por churn evitable}: La incapacidad de identificar y actuar sobre patrones de comportamiento que preceden al churn resulta en la pérdida de clientes que podrían haberse retenido mediante intervenciones oportunas. En el modelo SaaS, donde el costo de adquirir un nuevo cliente es significativamente mayor que retener uno existente, cada cliente perdido representa no solo la pérdida de ingresos recurrentes sino también el desperdicio de la inversión inicial en adquisición.

\textbf{Oportunidades de upselling y cross-selling desaprovechadas}: Sin un entendimiento claro de qué características y comportamientos predicen la propensión al upgrade, la empresa pierde oportunidades sistemáticas de incrementar el valor promedio por cliente. Esto es particularmente crítico en el contexto de clientes que demuestran alto uso pero pagan tarifas relativamente bajas.

\textbf{Ineficiencia en la asignación de recursos de Customer Success}: La falta de priorización basada en datos resulta en que los recursos limitados del equipo de Customer Success se distribuyan de manera subóptima, dedicando tiempo similar a clientes con muy diferentes potenciales de retención y crecimiento.

\subsubsection{Impactos Operacionales}

\textbf{Sobrecarga del equipo de soporte}: La gestión reactiva genera un volumen mayor de tickets de soporte y consultas que podrían prevenirse mediante la identificación temprana de problemas de adopción y el desarrollo de intervenciones proactivas personalizadas por segmento.

\textbf{Deterioro en la experiencia del cliente}: Los clientes que no reciben el nivel adecuado de atención o las comunicaciones relevantes para su etapa del ciclo de vida y patrón de uso experimentan una menor satisfacción, lo que se traduce en menor engagement, menor likelihood de renovación y menor propensión a recomendar el producto.

\subsubsection{Impactos Estratégicos a Largo Plazo}

\textbf{Pérdida de ventaja competitiva}: En el dinámico mercado HealthTech latinoamericano, donde la competencia incluye tanto startups ágiles como actores internacionales con recursos sustanciales, la incapacidad de optimizar la relación con la base de clientes existente puede resultar en pérdida de participación de mercado frente a competidores que sí implementen estrategias de customer success basadas en datos.


\section{Descripción y justificación del proyecto}
\subsection{En qué consiste el proyecto}
% Explicar el contenido del proyecto en el que participa el estudiante.

El proyecto titulado \emph{“Generar políticas de marketing para los clientes de RESERVO.cl, basado en el análisis de su comportamiento”} consiste en diseñar e implementar un marco de trabajo que permita:
\begin{enumerate}
  \item Recopilar y unificar datos de uso de funcionalidades, actividad de agendamiento, respuestas NPS y métricas financieras de cada cliente.
  \item Aplicar técnicas de segmentación avanzada y análisis predictivo para clasificar a los clientes según riesgo de churn, nivel de adopción y propensión a migrar de plan.
  \item Definir y validar un conjunto de políticas de marketing automatizadas (campañas de reenganche, ofertas de upsell, comunicaciones personalizadas) adaptadas a cada segmento y etapa del ciclo de vida.
\end{enumerate}

\subsection{Por qué se hace el proyecto}
La iniciativa responde directamente al problema identificado de \textbf{aprovechamiento insuficiente de la información de comportamiento}, que impide acciones proactivas de retención y expansión. Las razones clave son:
\begin{itemize}
  \item \textbf{Impacto financiero:} Retener clientes en riesgo reduce el churn evitable y mejora el ingreso recurrente; aprovechar oportunidades de upsell aumenta el valor promedio por contrato.
  \item \textbf{Estrategia de diferenciación:} En un mercado HealthTech competitivo, las empresas que ofrezcan comunicaciones y ofertas altamente personalizadas ganan ventaja frente a quienes aplican tácticas genéricas.
  \item \textbf{Eficiencia operativa:} Automatizar la segmentación y las campañas libera recursos del equipo de Customer Success y Marketing, permitiéndoles enfocarse en intervenciones de alto impacto.
  \item \textbf{Alineación con políticas internas:} Las directrices de RESERVO para impulsar la adopción y engagement de funcionalidades requieren un respaldo analítico que integre datos de uso, satisfacción y finanzas, tal como plantea la estrategia de la gerencia de Customer Success y el CEO.
\end{itemize}

\subsection{Cómo se planea abordar}
De forma general, el proyecto se desarrollará en tres fases:
\begin{enumerate}
  \item \textbf{Preparación de datos y modelado:} Extracción, limpieza y consolidación de fuentes internas (base de usuarios, logs de uso, tickets de soporte, NPS y facturación) en un data warehouse.
  \item \textbf{Análisis y segmentación:} Implementación de algoritmos de clustering y modelos de scoring predictivo para generar segmentos de riesgo y de oportunidad de upsell.
  \item \textbf{Definición de políticas y prueba piloto:} Diseño de plantillas de campañas y protocolos de contacto basados en los segmentos; ejecución de pilotos controlados para validar efectividad antes de la automatización completa.
\end{enumerate}

El detalle metodológico (herramientas, técnicas estadísticas y arquitectura de datos) se presenta en la sección de Metodología.

\subsection{Por qué se hace el proyecto}
% Justificar con razones de la organización (fuentes internas, estrategias, políticas). 
% Ir más allá de argumentos jerárquicos.
\subsection{Cómo se planea abordar (visión general)}
% Describir de forma general el enfoque; el detalle quedará para Metodología.

\section{Objetivos del proyecto}
\subsection{Objetivo general}
% Formular con verbo en infinitivo. Debe comunicar: propósito, producto generado y solicitante/beneficiarios.
\subsection{Objetivos específicos}
% Listar resultados/entregables parciales relevantes. No describir etapas ni actividades.

\section{Rol del estudiante}
\subsection{Responsabilidad principal}
% Describir el subproyecto o responsabilidad específica del estudiante.
\subsection{Contribución al proyecto global}
% Explicar por qué su rol es necesario para resolver el problema o desarrollar la oportunidad.

\section{Cronograma}
\subsection{Actividades principales}
% Definir actividades del proyecto (no tareas menores).
\subsection{Tiempos estimados}
% Indicar duración/fechas. Puede representarse luego como tabla o carta Gantt.

\section{Metodología (opcional)}
\subsection{Enfoque general y fases}
% Describir cómo se realizará/realiza el proyecto y sus grandes fases o procesos.
\subsection{Herramientas de investigación}
% Principales herramientas; ventajas y desventajas. Detallar solo si son novedosas o poco comunes.
\subsection{Recursos e insumos}
% Datos existentes, soluciones previas o paralelas e insumos usados; explicar su rol.

\section{Referencias y bibliografía principal}
% Listar fuentes usadas en Antecedentes y Justificación, y referencias científicas del informe.
% Usar formato APA 7 para citas y bibliografía.


% ------------------------------------------------------------------------------
% ANEXO
% ------------------------------------------------------------------------------
\begin{appendixd}

	\section{Estados del cliente}

    \subsection{}

	% Imagen, se numerará automáticamente con la letra del anexo según
	% la configuración \appendixindepobjnum
	\insertimage[\label{img:etapa_adquisicion}]{imagenes/etapa_adquisicion.png}{scale=0.70}{Etapa de Adquisición en Reservo}

    \subsection{}
    \insertimage[\label{img:etapa_venta}]{imagenes/etapa_venta.png}{scale=0.70}{Etapa de Venta en Reservo}

    \subsection{}
    %insertar imagen onboarding
    \insertimage[\label{img:etapa_onboarding}]{imagenes/etapa_onboarding.png}{scale=0.70}{Etapa de Onboarding del cliente en Reservo}

    \subsection{}
    %insertar imagen cliente activo
    \insertimage[\label{img:etapa_cliente_activo}]{imagenes/etapa_cliente_activo.png}{scale=0.80}{Ciclo de cliente activo en Reservo}



\end{appendixd}

% --- Fin cuerpo_informe.tex ---
